\fancyhead[LH]{复旦大学软件工程}
\fancyhead[RH]{第三章\quad 需求分析}
\section{需求分析}
\subsection{功能性需求}

%\begin{figure}[htb] 
%\center{\includegraphics[width=0.95\textwidth]  {fig/fig2.png}} 
%\bicaption{内热源沿径向的分布}{Distribution of internal heat sources along %the radial direction}
%\end{figure}

\subsubsection{信息管理系统}

\paragraph{志愿者信息管理}~{}
\\

志愿者可以在Volunet志愿服务系统上进行注册,管理员可以对志愿者信息进行管理。志愿者信息管理包括以下内容:

(1)志愿者信息管理:包括志愿者的基本信息、联系方式、参与的志愿服务项目和时长等信息的管理,以及管理员对志愿者账号和信息的审核、修改和删除等。

(2)项目信息管理:包括志愿服务项目的基本信息、服务内容、服务时间、地点和所需人数等信息的管理,以及管理员对项目的发布、审核、修改和删除等。

(3)数据统计和分析:通过Volunet志愿服务系统对志愿者参与的各项服务数据进行统计和分析,包括服务时长、服务次数、服务范围、服务对象等,以便管理员进行数据分析和决策。

(4)消息通知管理:Volunet志愿服务系统可以实现对志愿者和管理员的消息通知管理,包括发布通知、发送短信和邮件提醒等功能,以方便管理员与志愿者之间的沟通和信息交流。

(5)安全管理:Volunet志愿服务系统需要实现安全管理措施,包括用户身份验证、数据加密和备份、系统运行监控和异常处理等,以确保系统运行的安全性和稳定性。

\paragraph{志愿服务团队入驻与管理机制}~{}
\\

(1)入驻申请:志愿服务团队需要向Volunet志愿服务系统管理员提交入驻申请,包括团队的基本信息、成员构成、服务内容和服务区域等。


(2)审核管理:系统管理员对志愿服务团队的入驻申请进行审核和管理,以确保团队的合法性、规范性和质量。

(3)团队管理:系统管理员对入驻的志愿服务团队进行管理,包括团队信息的维护、成员信息的管理、服务项目的发布和管理等。

(4)绩效考核: Volunet志愿服务系统对志愿服务团队的服务绩效进行考核和评估,以确保团队服务水平的提高和服务质量的保障。

(5)培训支持:Volunet志愿服务系统可以为志愿服务团队提供培训支持,包括志愿服务知识、技能和管理等方面的培训,以提高团队成员的服务能力和管理水平。

(6)激励机制:Volunet志愿服务系统可以为志愿服务团队设置激励机制,包括荣誉证书、奖励金和服务时长统计等,以激发团队成员的服务热情和积极性。

(7)招募机制:志愿服务团队可以在Volunet志愿服务系统发起招募令,允许所有已经注册的志愿者通过平台报名并上传简历,并且平台自动将该志愿者在平台上发生的服务信息更新到志愿服务团队的后台。志愿服务团队可以根据志愿者相关信息选择是否添加其为团队成员。


\paragraph{对接政府部门志愿系统}~{}
\\

Volunet志愿服务系统通过开发,将志愿者服务系统与有关政府部门志愿平台系统进行对接,实现数据交互、信息传输等功能,提供更加便捷高效的志愿服务。

(1)志愿者在Volunet志愿服务平台参与的志愿时长,与当地志愿服务信息相同步认证,其志愿时长同等有效计入当地志愿服务系统。

(2)有关政府部门志愿平台的相关项目同步接入Volunet志愿服务系统,集成多平台信息资源,帮助志愿者获取更丰富的志愿服务信息。


\subsubsection{志愿服务系统}

(1)Volunet提供多样化、个性化的志愿服务项目参与方式:

$\bullet$ 分类导航搜索:用户可以根据服务地点、服务时间段、志愿服务团体、志愿服务项目类型等筛选所感兴趣的项目。

$\bullet$ 智能推荐:根据用户的个人信息(地点、年龄等)以及用户的过往参与经历,向用户推荐其更可能感兴趣参与的项目,减少用户检索查询的时间。

(2)活动报名:用户在选择自己感兴趣的项目之后,提交报名表(根据用户个人信息以及过往志愿服务经历一键匹配报名表单,用户可根据自己偏好进行微调)。活动主办方根据报名表选择合适的志愿者之后,向志愿者发出邀请链接,志愿者可进入该项目的志愿群。

(3)过程监督:志愿者到达活动现场可通过手机打卡签到签退(包含定位功能避免虚假打卡),打卡签退后自动形成服务时长录入到后台该志愿者档案中。

(4)活动反馈:志愿者在结束该项目之后,可以对该志愿活动项目、志愿活动组织方进行匿名评价和打分,供其他志愿者参考。




\subsubsection{爱心捐助系统}

\paragraph{捐助计划}~{}
\\

(1)Volunet将允许获得相关资质认定的志愿团体发布捐款项目,包括特定项目筹款、月捐计划等。这将为志愿团体提供更多的筹款途径,并且也能够让用户更方便地了解和支持各种社会公益项目。

(2)所有用户(包括访客、志愿者、管理员等)都可以对自己感兴趣的项目进行捐款,提供帮助。用户可以选择捐款金额、捐款方式以及捐款时间等。捐款后,用户将获得相应的积分或勋章认证,鼓励用户积极参与志愿服务和公益事业,增强用户的归属感和荣誉感。

(3)在捐款之后,用户将实时获取该项目的反馈。志愿团体将定期更新项目进展情况,同时提供相应的捐款使用报告,让用户能够清楚地了解自己的捐款被用于何处,提高用户的信任感和满意度。

\paragraph{公益品售卖}~{}
\\

获取相关资质认证的志愿服务团队可以在爱心商城售卖相关爱心公益品,用户浏览公益文创产品到待缴费项目,即可进行在线支付。

(1)志愿服务团队上传相应公益文创产品名称、价格、类别(用户自己使用/捐赠给需要帮助的人),并经由系统审核。

(2)用户在爱心商城可以根据自己的需求和喜好进行挑选和购买。Volunet可以通过在线银行支付的方式进行支付。在支付过程中,交易信息首先会发送到银行方,然后用户页面转到银行支付平台上,用户在银行支付平台上输入卡号/密码进行支付,支付成功后转回系统。Volunet需要与银行进行定时对帐,每次对帐会读取成功的支付信息,并根据相应的交易日期和交易号更新本系统中的支付状态。只有在线支付成功(对帐成功)后才会核销公益品配送。

(3)为了确保透明和公正,志愿服务团队需要提供每件公益品售卖所得款项的使用方式及相关证明。若捐款由具体公益组织项目进行管理,则应提供该组织的名称、联系信息以及证明文件。若购买公益品直接寄发给需要帮助的人,则应提供收件人姓名、地址以及相关证明文件。如果有未能提供相应反馈及证明的志愿服务组织,将被取消售卖资格。

(4)志愿者可以凭借志愿服务时长的相应积分,享受在合作范围内的公益纪念品折扣,以感谢志愿者的志愿付出。


\paragraph{志愿者证书及公益品配送}~{}
\\

(1)Volunet志愿服务系统对志愿者所获得证书的信息进行管理。一个志愿者可以参与多个项目并获得相应的证书,证书分为两种类型,电子证书与实体证书。当志愿项目完成后,参与的志愿者均可获得电子证书,但只有当选择发货时,实体证书才会随之发放。在库存量不够时可以先创建采购需求,随后将若干个采购需求组成一个采购单进行采购。

(2)志愿者选择相应实体证书发货并支付运费后以及用户在爱心商城上购买的商品支付成功后,Volunet将通过志愿者或用户填写的收件人、电话和配送地址等信息,会自动生成配送单,交由仓库管理方进行处理,安排出货和配送。

(3)配送过程中,Volunet将接入包裹路线查询的API(菜鸟或所选快递公司自己提供),实时显示快递的位置,方便志愿者或用户获取包裹当前所在地址。

(4)在订单配送成功后,快递员将向用户提供确认信息,并请求用户完成此次订单。


\subsubsection{公益课程系统}

(1)培训形式:Volunet志愿服务系统首页上显示热门的志愿者培训,包括在线视频、在线课程、面对面培训和研讨会等,志愿者可以根据资讯标签搜索或分类浏览的方式获取所需的培训信息。

(2)培训内容:Volunet志愿服务系统可以提供多种培训内容,包括志愿服务知识、技能、管理和安全等方面的培训,以提高志愿者的服务能力和管理水平。

(3)培训考核:Volunet志愿服务系统可以通过在线测试、问卷调查和实践考核等方式,对志愿者的培训效果进行考核和评估,以确保培训的有效性和质量。

(4)技能认证:Volunet志愿服务系统可以提供志愿者技能认证,包括认证考试和证书颁发等,以确保志愿者的技能水平和服务质量的保障。

(5)培训支持:Volunet志愿服务系统可以为志愿者提供培训支持,包括培训材料、培训指导和培训反馈等,以帮助志愿者提高学习效果和服务质量。

(6)培训记录:系统可以记录志愿者的培训记录,包括培训时间、培训内容和培训成绩等,以方便管理员进行培训管理和志愿者服务评估。


\subsubsection{交流论坛系统}

志愿者交流论坛:Volunet为志愿者提供的社交与分享平台——志友圈。论坛分板块运作,允许志愿者进行评论,点赞,转发(功能上支持点赞长按一键三连)。

$\bullet$ 树洞:供志愿者通过文字自由交流,采取前端匿名。

$\bullet$ 问答:供志愿者通过文字提问,进行经验交流。

$\bullet$ 随手拍:供志愿者记录下志愿服务中的美好瞬间,上传照片到平台上。

$\bullet$ 志愿手记:允许志愿者上传志愿服务“手记”,记录自己的志愿服务日常,允许上传文字、图片、视频。

$\bullet$ 热榜:根据上述板块,筛选出点击量和点击率等指标靠前的帖子进行展示。

\subsubsection{志愿交友系统}
(1)寻找志愿好友:可以根据用户参与的志愿活动,推荐具有类似公益爱好的朋友,并允许双方查询对方信息并添加好友。

(2)双方实时聊天:不同于论坛,该聊天发送形式为私聊。同时,聊天操作简易,并提供多种输入方式,保证实时对话。
% 1.1 用户管理
% · 用户注册:支持新用户注册账户。
% · 用户登录:用户输入用户名和密码进行身份验证。
% · 用户信息管理:允许用户修改个人信息,如密码、昵称等。
% · 用户权限管理:根据用户角色分配不同权限,如管理员、普通用户等。

% 1.2 媒体资源管理
% · 媒体资源分类:对影片、音乐、图片等媒体资源进行分类管理。
% · 媒体资源检索:提供关键词搜索功能,方便用户查找媒体资源。
% · 媒体资源播放:支持多种格式的媒体资源播放。
% · 媒体资源收藏:允许用户将喜欢的媒体资源添加到收藏夹。

% 1.3 设备控制
% · 设备连接:支持与多种主流音视频设备的连接,如显示器、音响等。
% · 设备状态监测:实时监测设备运行状态,如播放状态、音量等。
% · 设备操作:允许用户对设备进行远程操作,如播放、暂停、音量调节等。

% 1.4 系统设置
% · 个性化设置:提供丰富的个性化设置选项,如主题、播放模式等。
% · 参数调整:允许用户调整系统参数,如亮度、音量等。




%\begin{table}[!htbp]
    \centering
    \bicaption{高频感应加热的基本参数}{Basic parameters of high frequency induction heating}
    \begin{tabular}{|c| c|c|c|}
    \hline
    感应频率 &感应发生器功率 & 工件移动速度  &感应圈与零件间隙\\
    (KHz)&($\% \times$80Kw) &(mm/min)  &(mm)\\
    \hline
    250 &88 &5900 &1.65\\
    \hline
    250 &88 &5900 &1.65\\
    \hline
    250 &88 &5900 &1.65\\
    \hline
    250 &88 &5900 &1.65\\
    \hline
    \end{tabular}
\end{table}


\begin{table}
    \centering
    \captionsetup{singlelinecheck=off}
    \caption*{续表} %取消编号
    \begin{tabular}{|c| c|c|c|}
    \hline
    感应频率 &感应发生器功率 & 工件移动速度  &感应圈与零件间隙\\
    (KHz)&($\% \times$80Kw) &(mm/min)  &(mm)\\
    \hline
    250 &88 &5900 &1.65\\
    \hline
    250 &88 &5900 &1.65\\
    \hline
    \end{tabular}
\end{table}
%表格太大需要转页时,需要在续表上方注明“续表”,表头也应重复排出。


\subsection{非功能性需求}

\subsubsection{性能需求}

% 2.2 性能

 $\bullet   \enspace$ 响应速度:系统响应迅速,保证流畅的用户体验。 \hfill 
 
 $\bullet   \enspace$ 资源占用:合理利用系统资源,降低设备负担。 
  \hfill   
 
 $\bullet   \enspace$ 稳定性:系统运行稳定,减少故障发生。 
 \hfill  

\subsubsection{用户或人的因素}

% 2.1 易用性
 $\bullet   \enspace$ 界面设计:系统界面简洁、美观,易于操作。 \hfill 
 
 $\bullet   \enspace$ 操作指引:提供清晰的操作指南和帮助文档。 
  \hfill   
 
 $\bullet   \enspace$ 个性化定制:支持用户根据个人喜好进行界面和功能设置。
 \hfill  


\subsubsection{环境需求}

$\bullet   \enspace$ 服务器:Volunet志愿服务系统需要一个运行服务器来存储和处理数据,处理用户请求和提供服务。同时,服务器的性能应该足够强大,以便能够处理大量的用户请求和数据。
 \hfill  
 
 $\bullet   \enspace$ 数据库:Volunet志愿服务系统需要一个数据库来存储用户和组织等的信息,以及志愿活动的相关信息。此外,数据库应该能够支持高并发、高可用性和高性能的访问。
 \hfill  
 
 $\bullet   \enspace$ 网络:Volunet志愿服务系统需要一个稳定的网络环境,以便用户可以顺畅地访问系统和提交志愿服务申请。而且,网络环境应该能够支持高并发和高速的数据传输。
 \hfill  
 
 %$\bullet   \enspace$ 安全性:Volunet志愿服务系统需要具备一定的安全性能,以保护用户和组织的信息不被未经授权的人员访问和窃取。一般来说,系统应该采用安全的协议和技术,如SSL、加密等。
  $\bullet   \enspace$ 硬件设备支持: Volunet的签到签退功能需要用户设备(手机)运行地址定位功能,根据地址定位避免用户虚假打卡的情况发生。
 \hfill  
 
 %$\bullet   \enspace$ 用户界面:Volunet志愿服务系统需要一个友好、易用的用户界面,以便用户可以方便地浏览和提交志愿服务申请。简而言之,界面应该简洁明了,易于导航和操作。
 %\hfill  
 
 $\bullet   \enspace$ 技术支持与运行维护:Volunet志愿服务系统需要有专业的技术支持和运维团队,以便及时解决用户在使用系统中遇到的问题和故障。技术支持团队应该具备丰富的技术经验和专业知识,能够快速有效地响应和处理增加和修改的需求。
 \hfill  
 
\subsubsection{界面需求}
 $\bullet   \enspace$ 简洁明了:避免过多的文字和图标,使用户能够快速地了解Volunet志愿服务系统的功能和操作。
 \hfill 
 
 $\bullet   \enspace$ 易于导航:用户可以方便地浏览和查找各种志愿服务活动和组织,并快速地提交志愿服务申请。
 \hfill 
 
 $\bullet   \enspace$ 可定制性:用户可以根据自己的需求和喜好来自定义界面的颜色、字体、背景等。
 \hfill 
 
 $\bullet   \enspace$ 响应式设计:能够适应不同的设备和屏幕尺寸,如手机、平板电脑和电脑等。
 \hfill 
 
 $\bullet   \enspace$ 可访问性:能够满足老年用户等弱势群体的需求和使用体验,如提供语音提示、放大功能等。
 \hfill 
 
 $\bullet   \enspace$ 友好的交互体验:如提供实时反馈、动画效果等,以便增强用户的参与感和满意度。
 \hfill 
 
 %$\bullet   \enspace$ 安全性:保护用户的个人信息和数据不被未经授权的人员访问和窃取,界面可采用安全的协议和技术,如SSL、加密等。
 %\hfill 
 
\subsubsection{文档需求}
 $\bullet   \enspace$ 需求文档:详细描述Volunet志愿服务系统的功能和性能需求,以便开发团队能够根据需求进行开发和测试。
 \hfill 
 
 $\bullet   \enspace$ 设计文档:详细描述Volunet志愿服务系统的架构和设计方案,以便开发团队能够根据设计进行开发和测试。
 \hfill 
 
 $\bullet   \enspace$ 用户手册:详细描述Volunet志愿服务系统的功能和操作方法,以便用户能够快速地了解系统的使用方法和注意事项。
 \hfill 
 
 $\bullet   \enspace$ 管理手册:详细描述Volunet志愿服务系统的管理方法和流程,以便管理员能够快速地了解系统的管理方法和注意事项。
 \hfill 
 
 $\bullet   \enspace$ 测试文档:详细描述Volunet志愿服务系统的测试方法和流程,以便测试团队能够根据测试文档进行测试和验证。
 \hfill 
 
 $\bullet   \enspace$ 维护文档:详细描述Volunet志愿服务系统的维护方法和流程,以便维护团队能够快速地了解系统的维护方法和注意事项。
 \hfill 
 
 $\bullet   \enspace$ API文档:详细描述Volunet志愿服务系统的API接口和参数,以便其他开发者能够根据API文档进行系统的二次开发和集成。
 \hfill 
 
 $\bullet   \enspace$ 安全文档:详细描述Volunet志愿服务系统的安全策略和措施,以便管理员和用户能够了解系统的安全性和注意事项。
 \hfill 
 
\subsubsection{数据需求}

 $\bullet   \enspace$ 用户数据:包括各个用户所需的数据,如志愿者的个人信息、志愿服务记录、志愿服务时长等信息,以便Volunet志愿服务系统能够根据志愿者的需求和志愿服务记录进行匹配和推荐。
 \hfill 
 
% $\bullet   \enspace$ 组织数据:组织数据应该包括志愿团队在内的组织数据,比如基本信息、志愿服务活动信息、志愿服务记录等信息,以便Volunet志愿服务系统能够根据志愿团队的需求和志愿服务活动记录进行访问和加工。
 %\hfill 
 
 $\bullet   \enspace$ 活动数据:活动数据应该包括志愿服务活动在内活动信息,如基本信息、时间、地点、参与人数等信息。
 \hfill 
 
 $\bullet   \enspace$ 统计数据:统计数据应该包括系统的使用情况、用户的参与情况、志愿服务活动的参与情况等信息,以便系统能够进行数据分析和优化。
 \hfill 
 
 $\bullet   \enspace$ 日志数据:日志数据应该包括Volunet志愿服务系统的操作日志、错误日志、访问日志等信息,以便系统能够进行故障排除和性能优化。
 \hfill 
 
 $\bullet   \enspace$ 配置数据:配置数据应该包括Volunet志愿服务系统的配置信息、参数设置等信息,以便系统能够根据配置进行运行和管理。
 \hfill 
 
\subsubsection{资源使用需求}
$\bullet   \enspace$ 服务器资源:为运行和处理数据,服务器的性能应该足够强大,以便能够处理大量的用户请求和数据。需要注意的是,服务器的配置和数量应该根据系统的访问量和数据量进行调整。
\hfill 

$\bullet   \enspace$ 存储资源:Volunet志愿服务系统有存储用户和组织的信息,以及志愿服务活动的相关信息的需求,存储资源的容量和类型应该根据系统的数据量和类型进行调整。
\hfill 

$\bullet   \enspace$ 带宽资源:Volunet志愿服务系统需要足够的带宽资源来保证用户能够顺畅地访问系统和提交志愿服务申请,带宽资源的大小应该根据系统的访问量和数据传输量进行调整。
\hfill 

$\bullet   \enspace$ 软件资源:Volunet志愿服务系统需要一定的软件资源来运行和管理系统,如操作系统、数据库、Web服务器、应用服务器等。此外,软件资源的版本和类型应该根据系统的需求进行选择和配置。
\hfill 

$\bullet   \enspace$ 人力资源:Volunet志愿服务系统需要相应的人力资源来维护和管理系统,包括开发团队、测试团队、运维团队、客服团队等。相同的,人力资源的数量和技能水平应该根据系统的规模和需求进行配置和管理。
\hfill 

\subsubsection{安全保密需求}

% 2.3 安全性
$\bullet   \enspace$ 数据保护:对用户数据进行加密处理,确保信息安全。
\hfill 

$\bullet   \enspace$ 访问控制:通过权限管理防止未授权访问。
\hfill 

$\bullet   \enspace$ 安全更新:及时修复漏洞,提高系统安全性。
\hfill 



\subsubsection{可靠性需求}

% 2.7 可靠性
$\bullet   \enspace$ 异常处理:妥善处理各种异常情况,避免系统崩溃。
\hfill 

$\bullet   \enspace$ 数据备份:定期备份系统数据,防止数据丢失。
\hfill 

$\bullet   \enspace$ 容错能力:系统具备一定的容错能力,保证在部分模块出现故障时仍能正常运行。
\hfill 



\subsubsection{软件成本消耗需求}
$\bullet   \enspace$ 人力成本:Volunet志愿服务系统的开发和维护需要对应的人力成本,包括开发团队、测试团队、运维团队、客服团队等。
\hfill 

$\bullet   \enspace$ 软件成本:Volunet志愿服务系统的开发和维护需要一定的软件成本,包括开发工具、测试工具、运维工具等。
\hfill 

$\bullet   \enspace$ 硬件成本:Volunet志愿服务系统的运行需要足够的硬件成本,包括服务器、存储设备、网络设备等。
\hfill 

\subsubsection{其他非功能性需求}\\

(1) 兼容性

$\bullet   \enspace$ 操作系统支持:兼容不同的操作系统,如Windows、MacOS、Linux等。
\hfill 

$\bullet   \enspace$ 浏览器支持:支持不同的浏览器访问,如Chrome、Firefox、Safari、Edge等。
\hfill 

$\bullet   \enspace$ 分辨率支持:兼容不同的屏幕分辨率,以便能够适应不同的设备和屏幕尺寸。
\hfill 

$\bullet   \enspace$ 设备支持:支持不同的设备访问,如PC、手机、平板电脑等。
\hfill 

$\bullet   \enspace$ 标准兼容性:遵循标准的Web技术规范,如HTML、CSS、JavaScript等,以便能够在不同的浏览器和设备上正确地显示和运行。
\hfill 

$\bullet   \enspace$ 版本兼容性:考虑到不同的浏览器和设备的版本兼容性,特别是在新版本发布时,应该及时进行测试和优化。
\hfill \\

(2) 可扩展性

$\bullet   \enspace$ 模块化设计:采用模块化设计,便于功能扩展和升级。
\hfill 

$\bullet   \enspace$ 接口规范:提供统一的接口规范,方便与其他系统集成。
\hfill \\

(3) 可维护性

$\bullet   \enspace$ 代码规范:遵循编程规范,保证代码质量。
\hfill 

$\bullet   \enspace$ 文档完善:编写详细的设计文档、注释和用户手册,方便维护和使用。
\hfill 

$\bullet   \enspace$ 更新策略:制定合理的系统更新策略,持续改进和优化。
\hfill \\

(4) 可集成性

$\bullet   \enspace$ 标准化接口:提供标准化的接口,以便其他系统和应用程序能够通过API接口来访问和使用系统的数据和功能。
\hfill 

$\bullet   \enspace$ 文件导入导出:Volunet志愿服务系统应该提供文件导入导出功能,以便其他系统和应用程序能够通过文件导入导出来访问和使用系统的数据。
\hfill 

$\bullet   \enspace$ 集成插件:Volunet志愿服务系统应该提供集成插件,以便其他系统和应用程序能够通过集成插件来访问和使用系统的数据和功能。
\hfill 

$\bullet   \enspace$ 协议兼容性:Volunet志愿服务系统应该考虑到不同的协议兼容性,如SOAP、REST等,以便其他系统和应用程序能够通过不同的协议来访问和使用系统
\hfill 

%\vspace{-10mm}
%\begin{eqnarray}
%\frac{1}{\mu} \nabla^2A - j \omega \sigma A -\nabla(\frac{1}{\mu}) %\times(\nabla \times A)+J_0=0
%\end{eqnarray}

%\subsection{本章小结}
%\begin{figure}[htb] 
%    \center{\includegraphics[width=0.95\textwidth]  {fig/fig2.png}} 
%    \bicaption{内热源沿径向的分布}{Distribution of internal heat sources %along the radial direction}
%\end{figure}
% 本章介绍了……

\newpage