
\fancyhead[LH]{复旦大学软件工程}
\fancyhead[RH]{致\qquad 谢}

\addcontentsline{toc}{section}{致\qquad 谢}
\section*{致\qquad 谢}

\hspace{8mm}

在这份“Volunet志愿服务系统面向对象分析和设计文档”的书写过程中,我们团队得到了很多的支持和帮助,在此我们向所有对文档书写工作有贡献和提供帮助的人士表示由衷的感谢。

首先,我们要感谢著名的牛老师,在“软件工程”课程上为我们传授了许多和面向对象分析和设计相关的专业知识,并着重指出了文档书写中可能出现的误区和错误,这大大节省了我们分析项目和书写文档的时间。同时,牛老师匠心独运设计的课程项目让我们能够在实践中不断地提升自己,还让我们更好地理解和掌握了团队合作的重要性和技巧,领略到了技术以外的工程智慧。最后要特别感谢牛老师的八字真言,“逐层分解,逐步求精”,这在我们分析和完成项目任务时感到受用无穷。

其次,我们要感谢两位亲爱的助教老师,他们在整个课程中给予了我们很多的帮助和支持。他们不仅在课堂上为我们提供相关例题并解答疑惑,还在团队合作过程中给予了我们很多的建议和指导,让我们能够更好地协作完成任务。此外,要特别感谢助教老师提供的模板文档和参考资料,为最初身处迷雾之中的我们指明了优秀文档书写的方向。

最后,我们要感谢我们每个团队成员,没有大家互相之间的支持和配合,我们的“Volunet志愿服务系统面向对象分析和设计文档”的书写也不可能如此顺利地完成。在每节“软件工程”课程前,大家总会聚在一起,开一个项目的“早餐会”,总结上周的工作并讨论和分配下一周的任务。正是有大家上述的通力配合和项目智慧,我们的项目才能有条不紊地推进下去,并能最终按时提交一份保质保量的文档。

