\section{项目介绍}
\subsection{引言}
随着社会的进步和发展,越来越多的人开始关注社会责任和公益事业,积极参与志愿服务活动奉献爱心。然而,当下的志愿团队项目大多以原子化的方式分散,志愿团队需要花费精力增长维护自己的志愿社群,志愿者也需要加入无数个对应志愿社群来获取志愿项目信息,志愿服务的效果和效率并不理想。我们希望建立一个全链路、一站式的志愿服务平台Volunet,整合资源,赋能服务,帮助志愿者更好地参与、志愿团队更好地组织管理志愿服务,让志愿将我们彼此联结。

区别于现有单个志愿团队开发的独立系统和由政府牵头组织的中国志愿服务网 \footnote{\href{https://chinavolunteer.mca.gov.cn/site/home}{https://chinavolunteer.mca.gov.cn/site/home}},Volunet有以下突出优势:

\begin{itemize}[itemsep=2pt,topsep=2pt,parsep=4pt,itemindent=2em]
    \item \textbf{以人为中心的系统:} 围绕“志愿者”和“志愿团队”两大用户向度针对性定制个性化服务。对于志愿者,系统提供个性化用户画像,推荐和交友空间等服务;对于志愿团队,系统提供团队展示主页,进度管理反馈控制等服务。
    \item \textbf{科技向善,让所有爱心被妥善安放:} Volunet提供全链路一站式的志愿组件,包括公益课程、公益小店、捐助计划等。通过互联网技术使得进度可追溯,过程可监督的,形成全透明的志愿服务系统,让所有爱心被妥善安放。
    \item \textbf{用志愿服务重新定义本地生活:} 通过“志友圈”,志愿者可以对志愿活动进行点评反馈分享,也可以与其他志愿者聊天交友,让志愿服务成为了一种可以分享和发展的交友方式,促进社区的发展和建设。\\
\end{itemize}


本文档主要内容分为:开发规划、需求分析、面向对象分析和面向对象设计。在实现计划中,我们将列出系统的开发和测试等主要节点的计划;在需求分析中,我们将详细说明系统的功能需求和非功能需求;


\subsection{软件项目约束}

\begin{itemize}[itemsep=2pt,topsep=0pt,parsep=4pt,itemindent=1em]
    \item \textbf{时间约束:} 本项目的开发和实现需要在2023年春季学期“软件工程”课程项目提交与展示前完成。
    \item \textbf{资源约束:} 本项目的开发和实现需要在预算范围内进行,包括硬件、软件、人力和其他相关资源。
    \item \textbf{技术约束:} 本项目需要使用现代化的技术和工具进行开发和实现,包括但不限于Web低代码开发技术、数据库技术、服务器技术等。
    \item \textbf{版本控制约束:} 本项目需要使用版本控制工具进行管理和维护,确保代码的可追溯性和可维护性。
    \item \textbf{安全约束:} 本项目需要考虑数据的安全性和隐私保护,确保用户信息和数据不会被泄露或滥用。
    \item \textbf{可维护性约束:} 本项目需要考虑系统的可维护性和可扩展性,确保系统能够随着需求的变化进行升级和维护。
    \item \textbf{用户体验约束:} 本项目需要考虑用户的体验和使用便利性,确保系统的界面和操作流程符合用户的需求和习惯。
    \item \textbf{法律约束:} 本项目需要遵守相关法律法规和行业标准,确保系统的合法性和规范性。
\end{itemize}


以上是Volunet志愿服务系统的软件项目约束,我们将会在项目的开发和实现过程中严格遵守,确保项目的顺利完成并保证项目的开发质量。

\newpage