\fancyhead[LH]{复旦大学软件工程}
\fancyhead[RH]{第八章\quad 检验与测试}
\section{检验与测试}

\subsection{测试说明}
\subsubsection{测试目的}
\begin{itemize}[itemsep=2pt,topsep=0pt,parsep=4pt,itemindent=1em]
    \item 确定测试阶段的管理工作和技术工作提供参考,包括测试计划、测试用例、测试执行和测试报告等,以确保测试工作有序进行,测试结果可靠。
    \item 确定测试的内容和范围,包括系统功能测试、性能测试、安全测试、用户体验测试和兼容性测试,以覆盖系统的各个方面,发现潜在问题并及时解决。
    \item 为volunet志愿服务系统的评价提供依据和指标,包括用户注册、发布任务、接受任务、完成任务等功能的可用性、响应时间、并发用户数、数据处理速度、用户数据的保护、防止恶意攻击、防止数据泄露等安全性指标、界面设计、易用性、用户反馈等用户体验指标,以评估系统的优劣和提出改进建议。
    \item 发现系统的潜在问题并及时解决,提高系统的稳定性和用户体验,为用户提供更好的志愿服务体验。
\end{itemize}

\subsubsection{测试方针}
\begin{itemize}[itemsep=2pt,topsep=0pt,parsep=4pt,itemindent=1em]
    \item \textbf{测试目标:} 测试目标是发现系统的潜在问题并提供改进建议,以确保系统的稳定性和用户体验。
    \item \textbf{测试范围:} 测试范围包括系统功能测试、性能测试、安全测试、用户体验测试和兼容性测试,以覆盖系统的各个方面。
    \item \textbf{测试策略:} 测试策略是基于风险的测试,根据系统的重要性和影响程度,优先测试高风险区域和关键功能。
    \item \textbf{测试方法:} 测试方法包括手动测试和自动化测试,手动测试用于测试系统的人机交互和用户体验,自动化测试用于测试系统的功能和性能。
    \item \textbf{测试用例:} 测试用例是测试的基本单位,测试用例应该覆盖所有功能和场景,以确保系统的完整性和正确性。
    \item \textbf{测试数据:} 测试数据应该包括各种类型的数据,包括正常数据、异常数据、边界数据等,以确保系统的可靠性和鲁棒性。
    \item \textbf{测试环境:} 测试环境应该与生产环境相同或接近,以确保测试结果的可靠性和准确性。
    \item \textbf{测试报告:} 测试报告应该包括测试结果、测试覆盖率、测试缺陷和改进建议等,以便开发人员和项目管理人员了解测试进展和测试结果。
    \item \textbf{测试人员:} 测试人员应该具备专业的测试技能和知识,熟悉测试流程和测试工具,以确保测试的有效性和准确性。
\end{itemize}

\\

通过以上测试方针,可以确保测试工作的有效性和完整性,发现系统的潜在问题并及时解决,提高系统的稳定性和用户体验,为用户提供更好的志愿服务体验。
\subsection{功能测试}
\subsubsection{功能性测试说明}
\begin{itemize}[itemsep=2pt,topsep=0pt,parsep=4pt,itemindent=1em]
 \item \textbf{明确目标:} 测试应该有明确的目标和预期结果,以便能够确定测试是否成功。

\item \textbf{重要性:} 测试应该重点测试最重要的功能,以确保系统的核心功能正常工作。

\item \textbf{可重复性:} 功能性测试应该是可重复的,以便能够在需要时重复执行相同的测试,并获得相同的结果。

\item \textbf{可自动化:} 功能性测试应该是可自动化的,以便能够快速、准确地执行测试,并节省时间和资源。

\item \textbf{完整性:} 测试应该覆盖所有的功能,以确保系统的完整性。

\item \textbf{可测量:} 功能性测试应该是可测量的,以便能够确定测试的成功和失败。

\item \textbf{可维护性:} 功能性测试应该是可维护的,以便能够在系统更新后对测试进行更新和修改。

\item \textbf{可扩展性:} 功能性测试应该是可扩展的,以便能够在需要时添加新的测试用例。

\item \textbf{可靠性:} 功能性测试应该是可靠的,以便能够获得准确和可信的测试结果。

\item \textbf{可重构性:} 功能性测试应该是可重构的,以便能够在需要时对测试进行重构和改进。
\end{itemize}\\

限于篇幅,下面仅列出部分高优先级的功能性测试用例。

\subsubsection{部分功能性测试样例}
\begin{framed} \textbf{测试用例编号:} TC-A001

\textbf{测试模块:} 用户注册

\textbf{测试标题:} 注册新用户

\textbf{重要级别:} 高

\textbf{预置条件:} 无

\textbf{输入数据:} 用户名、密码、电子邮件地址

\textbf{预期输出:} 成功创建新用户并显示成功消息

\begin{center} \fbox{\parbox{0.8\textwidth}{TC-A001测试结果: 通过}} \end{center} \end{framed}

\begin{framed} \textbf{测试用例编号:} TC-A002

\textbf{测试模块:} 用户登录

\textbf{测试标题:} 使用正确的用户名和密码登录

\textbf{重要级别:} 高

\textbf{预置条件:} 已注册用户

\textbf{输入数据:} 用户名、密码

\textbf{预期输出:} 成功登录并显示欢迎消息

\begin{center} \fbox{\parbox{0.8\textwidth}{TC-A002测试结果: 通过}} \end{center} \end{framed}

\begin{framed} \textbf{测试用例编号:} TC-A003

\textbf{测试模块:} 用户登录

\textbf{测试标题:} 使用错误的用户名和密码登录

\textbf{重要级别:} 中

\textbf{预置条件:} 已注册用户

\textbf{输入数据:} 错误的用户名、密码

\textbf{预期输出:} 登录失败并显示错误消息

\begin{center} \fbox{\parbox{0.8\textwidth}{TC-A003测试结果: 通过}} \end{center} \end{framed}

\begin{framed} \textbf{测试用例编号:} TC-A004

\textbf{测试模块:} 用户管理

\textbf{测试标题:} 查看用户列表

\textbf{重要级别:} 中

\textbf{预置条件:} 已登录管理员账户

\textbf{输入数据:} 点击用户管理菜单

\textbf{预期输出:} 显示所有用户的列表

\begin{center} \fbox{\parbox{0.8\textwidth}{TC-A004测试结果: 通过}} \end{center} \end{framed}

\begin{framed} \textbf{测试用例编号:} TC-A005

\textbf{测试模块:} 用户管理

\textbf{测试标题:} 编辑用户信息

\textbf{重要级别:} 高

\textbf{预置条件:} 已登录管理员账户

\textbf{输入数据:} 选择一个用户并编辑他的信息

\textbf{预期输出:} 成功更新用户信息并显示成功消息

\begin{center} \fbox{\parbox{0.8\textwidth}{TC-A005测试结果: 通过}} \end{center} \end{framed}

\begin{framed} \textbf{测试用例编号:} TC-A006

\textbf{测试模块:} 用户管理

\textbf{测试标题:} 删除用户

\textbf{重要级别:} 高

\textbf{预置条件:} 已登录管理员账户

\textbf{输入数据:} 选择一个用户并删除他

\textbf{预期输出:} 成功删除用户并显示成功消息

\begin{center} \fbox{\parbox{0.8\textwidth}{TC-A006测试结果: 通过}} \end{center} \end{framed}

\begin{framed} \textbf{测试用例编号:} TC-A007

\textbf{测试模块:} 任务管理

\textbf{测试标题:} 创建任务

\textbf{重要级别:} 高

\textbf{预置条件:} 已登录管理员账户

\textbf{输入数据:} 任务标题、描述、截止日期等信息

\textbf{预期输出:} 成功创建任务并显示成功消息

\begin{center} \fbox{\parbox{0.8\textwidth}{TC-A007测试结果: 通过}} \end{center} \end{framed}

\begin{framed} \textbf{测试用例编号:} TC-A008

\textbf{测试模块:} 任务管理

\textbf{测试标题:} 查看任务列表

\textbf{重要级别:} 中

\textbf{预置条件:} 已登录用户账户

\textbf{输入数据:} 点击任务列表菜单

\textbf{预期输出:} 显示所有任务的列表

\begin{center} \fbox{\parbox{0.8\textwidth}{TC-A008测试结果: 通过}} \end{center} \end{framed}

\begin{framed} \textbf{测试用例编号:} TC-A009

\textbf{测试模块:} 活动发布

\textbf{测试标题:} 发布活动

\textbf{重要级别:} 高

\textbf{预置条件:} 用户已登录

\textbf{输入数据:} 活动名称、活动时间、活动地点、活动描述、招募人数

\textbf{预期输出:} 活动发布成功,并显示在活动列表中

\begin{center} \fbox{\parbox{0.8\textwidth}{TC-A009测试结果: 通过}} \end{center} \end{framed}

\begin{framed} \textbf{测试用例编号:} TC-A010

\textbf{测试模块:} 活动管理

\textbf{测试标题:} 修改活动

\textbf{重要级别:} 高

\textbf{预置条件:} 用户已登录,已发布一项活动

\textbf{输入数据:} 修改后的活动名称、活动时间、活动地点、活动描述、招募人数

\textbf{预期输出:} 活动信息更新成功,并显示在活动列表中

\begin{center} \fbox{\parbox{0.8\textwidth}{TC-A010测试结果: 通过}} \end{center} \end{framed}

\begin{framed} \textbf{测试用例编号:} TC-A011

\textbf{测试模块:} 活动管理

\textbf{测试标题:} 删除活动

\textbf{重要级别:} 高

\textbf{预置条件:} 用户已登录,已发布一项活动

\textbf{输入数据:} 删除该活动

\textbf{预期输出:} 活动信息删除成功,并不再显示在活动列表中

\begin{center} \fbox{\parbox{0.8\textwidth}{TC-A011测试结果: 通过}} \end{center} \end{framed}

\begin{framed} \textbf{测试用例编号:} TC-A012

\textbf{测试模块:} 入队管理

\textbf{测试标题:} 申请参加活动

\textbf{重要级别:} 高

\textbf{预置条件:} 用户已登录,已有一项活动发布

\textbf{输入数据:} 申请参加该活动

\textbf{预期输出:} 申请信息提交成功,并等待活动发布者审核

\begin{center} \fbox{\parbox{0.8\textwidth}{TC-A012测试结果: 通过}} \end{center} \end{framed}

\begin{framed} \textbf{测试用例编号:} TC-A013

\textbf{测试模块:} 活动管理

\textbf{测试标题:} 审核志愿者申请

\textbf{重要级别:} 高

\textbf{预置条件:} 用户已登录,已有一项待审核的志愿者申请

\textbf{输入数据:} 审核该志愿者申请,同意或拒绝

\textbf{预期输出:} 志愿者申请审核结果提交成功,并通知申请者审核结果

\begin{center} \fbox{\parbox{0.8\textwidth}{TC-A013测试结果: 通过}} \end{center} \end{framed}

\begin{framed} \textbf{测试用例编号:} TC-A014

\textbf{测试模块:} 活动管理

\textbf{测试标题:} 筛选活动

\textbf{重要级别:} 中

\textbf{预置条件:} 用户已登录,已发布多项活动

\textbf{输入数据:} 活动名称、活动时间、活动地点、招募人数等筛选条件

\textbf{预期输出:} 显示符合筛选条件的活动列表

\begin{center} \fbox{\parbox{0.8\textwidth}{TC-A014测试结果: 通过}} \end{center} \end{framed}
\subsubsection{测试通过条件}
\begin{itemize}[itemsep=2pt,topsep=0pt,parsep=4pt,itemindent=1em]
\item \textbf{错误率:}系统在给定输入的情况下,能够以较低的错误率通过所有的测试样例与场景。对于重要的功能,系统的错误率必须控制在极低的水平。例如,系统应该能够准确地将任务分配给正确的志愿者,确保任务的及时完成和质量。此外,系统应该能够确保志愿者和组织之间的所有通信都能够及时、准确地传递,以便志愿者能够了解任务需求并及时反馈任务进展情况。

\item \textbf{性能:}系统必须能够在合理的时间内响应用户的请求,并能够处理大量的数据和用户请求。例如,系统必须能够在几秒钟内响应用户提交的志愿申请,并能够同时处理多个申请。

\item \textbf{可靠性:}系统必须能够保持稳定的运行状态,并能够在出现故障时及时进行修复和恢复。例如,系统必须能够在服务器故障或网络中断时自动切换到备用服务器,并能够快速恢复服务。

\item \textbf{安全性:}系统必须能够保护用户的隐私和数据安全,并能够防止恶意攻击和数据泄露。例如,系统必须能够加密用户的个人信息和志愿服务记录,并能够检测和防止网络攻击和数据泄露。

\item \textbf{用户体验:}系统必须能够提供良好的用户体验,包括易于使用的界面、快速的响应时间、清晰的反馈信息等。例如,系统必须能够提供简单、直观的志愿服务申请界面,并能够及时反馈申请状态和志愿服务结果。

\end{itemize}

\subsection{健壮性测试}
\subsubsection{健壮性测试说明}
在软件开发过程中,由于开发时间、人力和物力等方面的限制,设计者很容易忽略系统的容错功能。然而,这也是导致软件健壮性差的一个主要原因,因此,健壮性测试是确保一个好的软件系统最终能够交付给用户的必要步骤。健壮性测试可以确保系统在各种异常情况下的稳定性和可靠性,同时提高系统的安全性、性能和可维护性,从而给用户带来更好的使用体验。因此,健壮性测试不应该被视为一项可有可无的任务,而应该被视为软件开发过程中不可或缺的一部分。\\
volunet志愿服务系统健壮性测试包括以下内容:
\begin{itemize}[itemsep=2pt,topsep=0pt,parsep=4pt,itemindent=1em]
\item 对关键进程或线程杀死,观察系统的行为是否正常。
\item 对关键进程或线程挂起,观察系统的行为是否正常。
\item 网络不通,观察系统行为是否正常。
\item 数据库不通,观察系统的行为是否正常。
\item 输入数据格式不符合要求,观察系统的行为是否正常。
\item 在系统中添加大量用户,以确定系统是否能够处理大量数据和用户流量。
\item 测试系统在长时间运行时的稳定性和可靠性,以确保系统能够持续运行。
\item 测试系统在出现错误时的响应能力,以确保系统能够正确处理错误并向用户提供有用的信息。
\end{itemize}
\subsubsection{部分健壮性测试样例}
\begin{framed} \textbf{测试用例编号:} TC-B001

\textbf{测试模块:} 用户注册

\textbf{测试标题:} 测试用户注册功能是否稳定

\textbf{测试内容:} 在注册页面中,输入有效的用户名和密码,然后点击“注册”按钮。在注册过程中,模拟网络中断、服务器崩溃、输入无效数据等情况,检查系统是否能够正确处理这些异常情况。

\textbf{重要级别:} 高

\textbf{预置条件:} 用户访问注册页面

\textbf{输入数据:} 有效的用户名和密码

\textbf{预期输出:} 注册成功页面或错误提示信息

\begin{center} \fbox{\parbox{0.8\textwidth}{TC-B001测试结果: 通过}} \end{center} \end{framed}

\begin{framed} \textbf{测试用例编号:} TC-B002

\textbf{测试模块:} 用户登录

\textbf{测试标题:} 测试用户登录功能是否稳定

\textbf{测试内容:} 在登录页面中,输入有效的用户名和密码,然后点击“登录”按钮。在登录过程中,模拟网络中断、服务器崩溃、输入无效数据等情况,检查系统是否能够正确处理这些异常情况。

\textbf{重要级别:} 高

\textbf{预置条件:} 用户访问登录页面

\textbf{输入数据:} 有效的用户名和密码

\textbf{预期输出:} 登录成功页面或错误提示信息

\begin{center} \fbox{\parbox{0.8\textwidth}{TC-B002测试结果: 通过}} \end{center} \end{framed}

\begin{framed} \textbf{测试用例编号:} TC-B003

\textbf{测试模块:} 活动发布

\textbf{测试标题:} 测试活动发布功能是否稳定

\textbf{测试内容:} 在活动发布页面中,输入有效的活动信息,然后点击“发布”按钮。在发布过程中,模拟网络中断、服务器崩溃、输入无效数据等情况,检查系统是否能够正确处理这些异常情况。

\textbf{重要级别:} 高

\textbf{预置条件:} 用户访问活动发布页面

\textbf{输入数据:} 有效的活动信息

\textbf{预期输出:} 活动发布成功页面或错误提示信息

\begin{center} \fbox{\parbox{0.8\textwidth}{TC-B003测试结果: 通过}} \end{center} \end{framed}

\begin{framed} \textbf{测试用例编号:} TC-B004

\textbf{测试模块:} 活动搜索

\textbf{测试标题:} 测试活动搜索功能是否稳定

\textbf{测试内容:} 在活动搜索页面中,输入有效的搜索关键字,然后点击“搜索”按钮。在搜索过程中,模拟网络中断、服务器崩溃、输入无效数据等情况,检查系统是否能够正确处理这些异常情况。

\textbf{重要级别:} 中

\textbf{预置条件:} 用户访问活动搜索页面

\textbf{输入数据:} 有效的搜索关键字

\textbf{预期输出:} 活动搜索结果页面或错误提示信息

\begin{center} \fbox{\parbox{0.8\textwidth}{TC-B004测试结果: 通过}} \end{center} \end{framed}
\begin{framed} \textbf{测试用例编号:} TC-B005

\textbf{测试模块:} 活动报名

\textbf{测试标题:} 测试活动报名功能是否稳定

\textbf{测试内容:} 在活动详情页面中,点击“报名”按钮,然后填写有效的报名信息,最后点击“提交”按钮。在报名过程中,模拟网络中断、服务器崩溃、输入无效数据等情况,检查系统是否能够正确处理这些异常情况。

\textbf{重要级别:} 高

\textbf{预置条件:} 用户访问活动详情页面

\textbf{输入数据:} 有效的报名信息

\textbf{预期输出:} 报名成功页面或错误提示信息

\begin{center} \fbox{\parbox{0.8\textwidth}{TC-B005测试结果: 通过}} \end{center} \end{framed}

\begin{framed} \textbf{测试用例编号:} TC-B006

\textbf{测试模块:} 活动取消报名

\textbf{测试标题:} 测试活动取消报名功能是否稳定

\textbf{测试内容:} 在活动详情页面中,点击“取消报名”按钮。在取消报名过程中,模拟网络中断、服务器崩溃等情况,检查系统是否能够正确处理这些异常情况。

\textbf{重要级别:} 中

\textbf{预置条件:} 用户已经报名了该活动

\textbf{输入数据:} 无

\textbf{预期输出:} 活动取消报名成功页面或错误提示信息

\begin{center} \fbox{\parbox{0.8\textwidth}{TC-B006测试结果: 通过}} \end{center} \end{framed}

\begin{framed} \textbf{测试用例编号:} TC-B007

\textbf{测试模块:} 用户个人信息修改

\textbf{测试标题:} 测试用户个人信息修改功能是否稳定

\textbf{测试内容:} 在用户个人信息页面中,修改有效的个人信息,然后点击“保存”按钮。在修改过程中,模拟网络中断、服务器崩溃、输入无效数据等情况,检查系统是否能够正确处理这些异常情况。

\textbf{重要级别:} 高

\textbf{预置条件:} 用户已经登录并访问了个人信息页面

\textbf{输入数据:} 有效的个人信息

\textbf{预期输出:} 个人信息修改成功页面或错误提示信息

\begin{center} \fbox{\parbox{0.8\textwidth}{TC-B007测试结果: 通过}} \end{center} \end{framed}

\begin{framed} \textbf{测试用例编号:} TC-B008

\textbf{测试模块:} 活动评价

\textbf{测试标题:} 测试活动评价功能是否稳定

\textbf{测试内容:} 在活动详情页面中,点击“评价”按钮,然后填写有效的评价信息,最后点击“提交”按钮。在评价过程中,模拟网络中断、服务器崩溃、输入无效数据等情况,检查系统是否能够正确处理这些异常情况。

\textbf{重要级别:} 中

\textbf{预置条件:} 用户已经报名了该活动

\textbf{输入数据:} 有效的评价信息

\textbf{预期输出:} 活动评价成功页面或错误提示信息

\begin{center} \fbox{\parbox{0.8\textwidth}{TC-B008测试结果: 通过}} \end{center} \end{framed}

\begin{framed} \textbf{测试用例编号:} TC-B009

\textbf{测试模块:} 活动举报

\textbf{测试标题:} 测试活动举报功能是否稳定

\textbf{测试内容:} 在活动详情页面中,点击“举报”按钮,然后填写有效的举报信息,最后点击“提交”按钮。在举报过程中,模拟网络中断、服务器崩溃、输入无效数据等情况,检查系统是否能够正确处理这些异常情况。

\textbf{重要级别:} 中

\textbf{预置条件:} 用户已经登录并访问了活动详情页面

\textbf{输入数据:} 有效的举报信息

\textbf{预期输出:} 活动举报成功页面或错误提示信息

\begin{center} \fbox{\parbox{0.8\textwidth}{TC-B009测试结果: 通过}} \end{center} \end{framed}

\begin{framed} \textbf{测试用例编号:} TC-B010

\textbf{测试模块:} 系统安全

\textbf{测试标题:} 测试系统安全性是否稳定

\textbf{测试内容:} 模拟网络攻击、SQL注入、XSS攻击等安全攻击,检查系统是否能够正确处理这些攻击并保护用户信息的安全。

\textbf{重要级别:} 高

\textbf{预置条件:} 无

\textbf{输入数据:} 恶意攻击数据

\textbf{预期输出:} 系统能够正确处理攻击并保护用户信息的安全

\begin{center} \fbox{\parbox{0.8\textwidth}{TC-B010测试结果: 通过}} \end{center} \end{framed}

\begin{framed} \textbf{测试用例编号:} TC-B011

\textbf{测试模块:} 活动管理

\textbf{测试标题:} 测试活动管理功能是否稳定

\textbf{测试内容:} 在活动管理页面中,管理员可以对活动进行管理,包括审核、取消、修改等操作。在进行操作过程中,模拟网络中断、服务器崩溃、输入无效数据等情况,检查系统是否能够正确处理这些异常情况。

\textbf{重要级别:} 高

\textbf{预置条件:} 管理员已经登录并访问了活动管理页面

\textbf{输入数据:} 有效的管理操作

\textbf{预期输出:} 操作成功页面或错误提示信息

\begin{center} \fbox{\parbox{0.8\textwidth}{TC-B011测试结果: 通过}} \end{center} \end{framed}

\begin{framed} \textbf{测试用例编号:} TC-B012

\textbf{测试模块:} 用户管理

\textbf{测试标题:} 测试用户管理功能是否稳定

\textbf{测试内容:} 在用户管理页面中,管理员可以对用户进行管理,包括禁言、封号、删除等操作。在进行操作过程中,模拟网络中断、服务器崩溃、输入无效数据等情况,检查系统是否能够正确处理这些异常情况。

\textbf{重要级别:} 高

\textbf{预置条件:} 管理员已经登录并访问了用户管理页面

\textbf{输入数据:} 有效的管理操作

\textbf{预期输出:} 操作成功页面或错误提示信息

\begin{center} \fbox{\parbox{0.8\textwidth}{TC-B012测试结果: 通过}} \end{center} \end{framed}

\begin{framed} \textbf{测试用例编号:} TC-B013

\textbf{测试模块:} 数据备份与恢复

\textbf{测试标题:} 测试数据备份与恢复功能是否稳定

\textbf{测试内容:} 在系统设置页面中,管理员可以进行数据备份与恢复操作。在进行操作过程中,模拟网络中断、服务器崩溃等情况,检查系统是否能够正确处理这些异常情况,并能够正确备份和恢复数据。

\textbf{重要级别:} 高

\textbf{预置条件:} 管理员已经登录并访问了系统设置页面

\textbf{输入数据:} 有效的备份和恢复操作

\textbf{预期输出:} 操作成功页面或错误提示信息

\begin{center} \fbox{\parbox{0.8\textwidth}{TC-B013测试结果: 通过}} \end{center} \end{framed}

\begin{framed} \textbf{测试用例编号:} TC-B014

\textbf{测试模块:} 系统性能

\textbf{测试标题:} 测试系统性能是否稳定

\textbf{测试内容:} 在高并发、大数据量等情况下,模拟多个用户同时访问系统,检查系统是否能够正确处理这些请求,并且能够保持系统的稳定性和响应速度。

\textbf{重要级别:} 高

\textbf{预置条件:} 无

\textbf{输入数据:} 大量的并发请求和数据

\textbf{预期输出:} 系统能够正确处理请求并保持稳定性和响应速度

\begin{center} \fbox{\parbox{0.8\textwidth}{TC-B014测试结果: 通过}} \end{center} \end{framed}

\begin{framed} \textbf{测试用例编号:} TC-B015

\textbf{测试模块:} 系统兼容性

\textbf{测试标题:} 测试系统兼容性是否稳定

\textbf{测试内容:} 在不同的操作系统、浏览器、设备上访问系统,检查系统是否能够正确显示和运行,并且能够保持稳定性和响应速度。

\textbf{重要级别:} 中

\textbf{预置条件:} 无

\textbf{输入数据:} 不同的操作系统、浏览器、设备

\textbf{预期输出:} 系统能够正确显示和运行,并保持稳定性和响应速度

\begin{center} \fbox{\parbox{0.8\textwidth}{TC-B015测试结果: 通过}} \end{center} \end{framed}
\subsubsection{测试通过条件}

根据提供的测试要求和条件,可以总结出volunet志愿服务系统的健壮性测试通过条件如下:
\begin{itemize}[itemsep=2pt,topsep=0pt,parsep=4pt,itemindent=1em]
\item 有关进程或线程的测试样例,预期输出不符合调教的概率低于万分之一。
\item 有关底层(网络、数据库等)的测试样例,请求预期输出不符合条件的概率低于千分之一。
\item 其余测试样例,请求预期输出不符合条件的概率低于十万分之一。
\item 所有测试样例都要重复运行10000次及以上。
\item 所有测试样例需要在真实环境下进行测试,模拟真实用户使用情况。
\item 在测试过程中,需要记录并分析系统出现的异常情况,如系统崩溃、数据丢失、响应时间过长等,以及异常情况的处理方式和效果。
\item 在测试完成后,需要对测试结果进行全面分析和总结,包括系统的性能、稳定性、可靠性等方面的评估,并提出改进建议和优化方案。
\end{itemize}
\subsection{压力测试}
\subsubsection{压力测试说明}
压力测试是给软件不断加压,强制其在超过正常运作以外的条件下运作系统,观察系统运行的状态,从而发现系统的缺陷或者评估系统的性能。
Volunet志愿服务系统的压力测试主要包括以下三项内容:
\begin{itemize}[itemsep=2pt,topsep=0pt,parsep=4pt,itemindent=1em]
\item \textbf{负载测试:}
负载测试:逐步增加系统负载,测试系统性能的变化,并最终确定在满足性能指标的情况下,系统所能承受的最大负载量。我们将模拟大量用户同时使用系统,例如同时进行注册、发布任务、查看任务等操作,以测试系统在高负载情况下的稳定性和响应速度。
\item \textbf{并发性能测试:}
通过逐渐增加并发用户数负载,直到系统的瓶颈或者不能接收的状态,综合分析执行指标来确定系统并发性能。我们将模拟多个用户同时进行相同或不同的操作,例如同时进行任务发布和任务申请,以测试系统在高并发情况下的稳定性和响应速度。
\item \textbf{疲劳强度测试:}
通过逐渐增加并发用户数负载,直到系统的瓶颈或者不能接收的状态。我们将模拟大量用户长时间使用系统,例如连续多小时进行任务发布、任务申请等操作,以测试系统在高强度使用情况下的稳定性和响应速度。
\end{itemize}
\subsubsection{部分压力测试样例}
\begin{framed} \textbf{测试用例编号:} TC-C001

\textbf{测试方式:} 并发性能测试

\textbf{场景:} 1000个用户同时登录系统

\textbf{测试内容:} 测试系统在1000个用户同时登录时的性能表现,包括响应时间、吞吐量、错误率等。

\textbf{重要级别:} 高

\textbf{性能指标:} 响应时间、吞吐量、错误率

\textbf{性能指标要求:} 响应时间不超过3秒,吞吐量不低于1000个请求/秒,错误率不超过0.5%

\begin{center}
\fbox{\parbox{0.8\textwidth}{TC-C001测试结果: 通过}}
\end{center}
\end{framed}

\begin{framed} \textbf{测试用例编号:} TC-C002

\textbf{测试方式:} 负载测试

\textbf{场景:} 500个用户同时进行志愿服务申请操作

\textbf{测试内容:} 测试系统在500个用户同时进行志愿服务申请操作时的性能表现,包括响应时间、吞吐量、错误率等。

\textbf{重要级别:} 高

\textbf{性能指标:} 响应时间、吞吐量、错误率

\textbf{性能指标要求:} 响应时间不超过5秒,吞吐量不低于500个请求/秒,错误率不超过1%

\begin{center}
\fbox{\parbox{0.8\textwidth}{TC-C002测试结果: 通过}}
\end{center}
\end{framed}

\begin{framed} \textbf{测试用例编号:} TC-C003

\textbf{测试方式:} 稳定性测试

\textbf{场景:} 持续运行24小时,模拟1000个用户进行各种操作

\textbf{测试内容:} 测试系统在持续运行24小时,模拟1000个用户进行各种操作时的稳定性表现,包括系统崩溃、数据丢失等情况。

\textbf{重要级别:} 高

\textbf{性能指标:} 系统稳定性

\textbf{性能指标要求:} 运行24小时内不出现系统崩溃、数据丢失等情况。

\begin{center}
\fbox{\parbox{0.8\textwidth}{TC-C003测试结果: 通过}}
\end{center}
\end{framed}

\begin{framed} \textbf{测试用例编号:} TC-C004

\textbf{测试方式:} 并发性能测试

\textbf{场景:} 1000个用户同时进行在线聊天

\textbf{测试内容:} 测试系统在1000个用户同时进行在线聊天时的性能表现,包括响应时间、吞吐量、错误率等。

\textbf{重要级别:} 中

\textbf{性能指标:} 响应时间、吞吐量、错误率

\textbf{性能指标要求:} 响应时间不超过5秒,吞吐量不低于800个请求/秒,错误率不超过1%

\begin{center}
\fbox{\parbox{0.8\textwidth}{TC-C004测试结果: 通过}}
\end{center}
\end{framed}

\begin{framed} \textbf{测试用例编号:} TC-C005

\textbf{测试方式:} 负载测试

\textbf{场景:} 500个用户同时进行在线视频会议

\textbf{测试内容:} 测试系统在500个用户同时进行在线视频会议时的性能表现,包括响应时间、吞吐量、错误率等。

\textbf{重要级别:} 中

\textbf{性能指标:} 响应时间、吞吐量、错误率

\textbf{性能指标要求:} 响应时间不超过10秒,吞吐量不低于300个请求/秒,错误率不超过2%

\begin{center}
\fbox{\parbox{0.8\textwidth}{TC-C005测试结果: 通过}}
\end{center}
\end{framed}

\begin{framed} \textbf{测试用例编号:} TC-C006

\textbf{测试方式:} 并发性能测试

 \textbf{场景:} 2000个用户同时进行志愿服务评价操作

\textbf{测试内容:} 测试系统在2000个用户同时进行志愿服务评价操作时的性能表现,包括响应时间、吞吐量、错误率等。

\textbf{重要级别:} 中

\textbf{性能指标:} 响应时间、吞吐量、错误率

\textbf{性能指标要求:} 响应时间不超过5秒,吞吐量不低于1000个请求/秒,错误率不超过1%

\begin{center} \fbox{\parbox{0.8\textwidth}{TC-C006测试结果: 通过}} \end{center} \end{framed}

\begin{framed} \textbf{测试用例编号:} TC-C007

\textbf{测试方式:} 稳定性测试

\textbf{场景:} 持续运行48小时,模拟5000个用户进行各种操作

\textbf{测试内容:} 测试系统在持续运行48小时,模拟5000个用户进行各种操作时的稳定性表现,包括系统崩溃、数据丢失等情况。

\textbf{重要级别:} 高

\textbf{性能指标:} 系统稳定性

\textbf{性能指标要求:} 运行48小时内不出现系统崩溃、数据丢失等情况。

\begin{center} \fbox{\parbox{0.8\textwidth}{TC-C007测试结果: 通过}} \end{center} \end{framed}


\subsubsection{测试通过条件}
Volunet志愿服务系统的压力测试通过条件如下:
\begin{itemize}
[itemsep=2pt,topsep=0pt,parsep=4pt,itemindent=1em]
\item \textbf{负载测试:}
负载测试:在逐步增加系统负载的过程中,系统的响应时间不应超过5秒,系统不应出现崩溃或错误等异常情况。在100次测试中,至少有99次测试结果满足性能指标要求。
\item \textbf{并发性能测试:}
并发性能测试:在逐渐增加并发用户数负载的过程中,系统的响应时间不应超过5秒,系统不应出现崩溃或错误等异常情况。在100次测试中,至少有99次测试结果满足性能指标要求。
\item \textbf{疲劳强度测试:}
疲劳强度测试:在长时间、高强度使用系统的过程中,系统的响应时间不应超过5秒,系统不应出现崩溃或错误等异常情况。在100次测试中,至少有99次测试结果满足性能指标要求。
\item \textbf{总体通过率:}
对于以上三种测试样例,系统至少需要满足99\%的测试结果符合性能指标要求,才能算通过压力测试。
\end{itemize}

\subsection{面向对象测试}
关于volunet志愿服务系统,本文档采用面向对象的方式设计,因此在测试时应该采用面向对象测试方法来进行测试。除了传统的测试方法,还应该选用类测试、类间测试和基于场景的测试方法进行测试。

类测试是测试单个类的功能和行为是否符合预期。在测试中,应该测试类的所有公共方法和属性,包括输入和输出参数、异常情况和边界条件等。例如,在volunet志愿服务系统中,可以测试志愿者类的报名、查询、修改、取消等方法是否正常工作。

类间测试是测试不同类之间的交互和协作是否正确。在测试中,应该测试类之间的消息传递、数据交换和调用关系等。例如,在volunet志愿服务系统中,可以测试志愿者类和活动管理类之间的交互是否正常。

基于场景的测试是测试系统在不同场景下的行为是否符合预期。在测试中,应该测试系统在不同的输入和环境条件下的行为和响应。例如,在volunet志愿服务系统中,可以测试志愿者报名活动、取消报名、查询活动等情况下的系统的行为。
\subsubsection{类测试}
\paragraph{类操作测试}~{}
\begin{framed} \textbf{测试用例编号:} TC-D001

\textbf{测试系统:} 信息管理系统

\textbf{测试类: } 志愿者类

\textbf{测试操作:} 添加志愿者

\textbf{重要级别:} 高

\textbf{测试场景:} 当管理员需要添加新的志愿者时

\textbf{测试内容:} 管理员在系统中添加新的志愿者,包括姓名、电话、电子邮件、地址等信息

\textbf{测试标准:} 新的志愿者信息能够被成功添加到系统中,管理员可以在系统中查看到新的志愿者信息

\begin{center} \fbox{\parbox{0.8\textwidth}{TC-D001测试结果: 通过}} \end{center} \end{framed}

\begin{framed} \textbf{测试用例编号:} TC-D002

\textbf{测试系统:} 信息管理系统

\textbf{测试类: } 志愿者类

\textbf{测试操作:} 更新志愿者信息

\textbf{重要级别:} 高

\textbf{测试场景:} 当志愿者需要更新个人信息时

\textbf{测试内容:} 志愿者在系统中更新个人信息,包括电话、电子邮件、地址等信息

\textbf{测试标准:} 志愿者的个人信息能够被成功更新到系统中,管理员可以在系统中查看到志愿者的最新信息

\begin{center} \fbox{\parbox{0.8\textwidth}{TC-D002测试结果: 通过}} \end{center} \end{framed}

\begin{framed} \textbf{测试用例编号:} TC-D003

\textbf{测试系统:} 信息管理系统

\textbf{测试类: } 志愿者类

\textbf{测试操作:} 删除志愿者

\textbf{重要级别:} 高

\textbf{测试场景:} 当管理员需要删除志愿者时

\textbf{测试内容:} 管理员在系统中删除某个志愿者的信息

\textbf{测试标准:} 被删除的志愿者的信息能够被从系统中成功删除,管理员无法在系统中查看到被删除的志愿者的信息

\begin{center} \fbox{\parbox{0.8\textwidth}{TC-D003测试结果: 通过}} \end{center} \end{framed}

\begin{framed} \textbf{测试用例编号:} TC-D004

\textbf{测试系统:} 志愿服务系统

\textbf{测试类: } 志愿项目类

\textbf{测试操作:} 创建新项目

\textbf{重要级别:} 高

\textbf{测试场景:} 当管理员需要创建新的志愿服务项目时

\textbf{测试内容:} 管理员在系统中创建新的志愿服务项目,包括项目名称、描述、所需志愿者数量等信息

\textbf{测试标准:} 新的志愿服务项目能够被成功创建并添加到系统中,管理员可以在系统中查看到新的志愿服务项目的信息

\begin{center} \fbox{\parbox{0.8\textwidth}{TC-D004测试结果: 通过}} \end{center} \end{framed}

\begin{framed} \textbf{测试用例编号:} TC-D005

\textbf{测试系统:} 志愿服务系统

\textbf{测试类: } 志愿项目类

\textbf{测试操作:} 更新项目信息

\textbf{重要级别:} 高

\textbf{测试场景:} 当管理员需要更新项目信息时

\textbf{测试内容:} 管理员在系统中更新项目信息,包括项目名称、描述、所需志愿者数量等信息

\textbf{测试标准:} 项目信息能够被成功更新到系统中,管理员可以在系统中查看到项目的最新信息

\begin{center} \fbox{\parbox{0.8\textwidth}{TC-D005测试结果: 通过}} \end{center} \end{framed}

 \begin{framed} \textbf{测试用例编号:} TC-D006

\textbf{测试系统:} 交流论坛系统

\textbf{测试类: } 帖子管理

\textbf{测试操作:} 发布帖子

\textbf{重要级别:} 高

\textbf{测试场景:} 用户在交流论坛系统中发布帖子

\textbf{测试内容:} 用户登录交流论坛系统,进入帖子发布页面 ,,输入帖子标题和内容,选择帖子分类,点击发布按钮,检查帖子是否成功发布

\textbf{测试标准:} 帖子成功发布,且在论坛中显示出来

\begin{center} \fbox{\parbox{0.8\textwidth}{TC-D006测试结果: 通过}} \end{center} \end{framed}

\begin{framed} \textbf{测试用例编号:} TC-D007

\textbf{测试系统:} 交流论坛系统

\textbf{测试类: } 帖子管理

\textbf{测试操作:} 编辑帖子

\textbf{重要级别:} 高

\textbf{测试场景:} 用户在交流论坛系统中编辑帖子

\textbf{测试内容:}用户登录交流论坛系统 找到要编辑的帖子 点击编辑按钮 修改帖子标题和内容 点击保存按钮 检查帖子是否成功编辑

\textbf{测试标准:} 帖子成功编辑,且在论坛中显示出来

\begin{center} \fbox{\parbox{0.8\textwidth}{TC-D007测试结果: 通过}} \end{center} \end{framed}

\begin{framed} \textbf{测试用例编号:} TC-D008

\textbf{测试系统:} 交流论坛系统

\textbf{测试类: } 帖子管理

\textbf{测试操作:} 删除帖子

\textbf{重要级别:} 高

\textbf{测试场景:} 用户在交流论坛系统中删除帖子

\textbf{测试内容:} 用户登录交流论坛系统 找到要删除的帖子 点击删除按钮 确认删除操作 检查帖子是否成功删除
\textbf{测试标准:} 帖子成功删除,且在论坛中不再显示

\begin{center} \fbox{\parbox{0.8\textwidth}{TC-D008测试结果: 通过}} \end{center} \end{framed}

\begin{framed} \textbf{测试用例编号:} TC-D009

\textbf{测试系统:} 交流论坛系统

\textbf{测试类: } 帖子管理

\textbf{测试操作:} 置顶帖子

\textbf{重要级别:} 中

\textbf{测试场景:} 管理员在交流论坛系统中置顶帖子

\textbf{测试内容:}  管理员登录交流论坛系统 找到要置顶的帖子 点击置顶按钮 检查帖子是否成功置顶

\textbf{测试标准:} 帖子成功置顶,且在论坛中显示在置顶位置

\begin{center} \fbox{\parbox{0.8\textwidth}{TC-D009测试结果: 通过}} \end{center} \end{framed}

 \begin{framed} \textbf{测试用例编号:} TC-D010

\textbf{测试系统:} 爱心捐助系统

\textbf{测试类: } 捐款者

\textbf{测试操作:} 捐款

\textbf{重要级别:} 高

\textbf{测试场景:} 捐款者已经登录到系统并选择了捐赠的项目

\textbf{测试内容:} 输入无效的捐款金额,如负数或非数字字符 输入合法的捐款金额,但超过了捐款者账户余额 输入合法的捐款金额,且不超过捐款者账户余额
\textbf{测试标准:} 系统应该提示错误信息并要求重新输入捐款金额 系统应该提示错误信息并要求重新输入捐款金额 系统应该成功处理捐款,并更新捐款者账户余额及项目筹集金额

\begin{center} \fbox{\parbox{0.8\textwidth}{TC-D010测试结果: 通过}} \end{center} \end{framed}

\begin{framed} \textbf{测试用例编号:} TC-D011

\textbf{测试系统:} 爱心捐助系统

\textbf{测试类: } 捐款者

\textbf{测试操作:} 查看个人捐款记录

\textbf{重要级别:} 中

\textbf{测试场景:} 捐款者已经登录到系统并捐赠了多项项目

\textbf{测试内容:}  查看个人捐款记录,记录为空 查看个人捐款记录,记录不为空 查看个人捐款记录,记录包含已取消的捐款

\textbf{测试标准:} 系统应该提示记录为空 系统应该正确显示捐款记录及相关信息 系统应该正确显示捐款记录及相关信息,包括已取消的捐款

\begin{center} \fbox{\parbox{0.8\textwidth}{TC-D011测试结果: 通过}} \end{center} \end{framed}

\begin{framed} \textbf{测试用例编号:} TC-D012

\textbf{测试系统:} 爱心捐助系统

\textbf{测试类: } 捐款者

\textbf{测试操作:} 取消捐款

\textbf{重要级别:} 高

\textbf{测试场景:} 捐款者已经登录到系统并捐赠了一项项目

\textbf{测试内容:} 选择不存在的捐款记录进行取消 选择已完成的捐款记录进行取消 选择未完成的捐款记录进行取消

\textbf{测试标准:} 系统应该提示错误信息并要求重新输入捐款记录 系统应该提示错误信息并要求重新输入捐款记录 系统应该成功取消捐款,并更新捐款者账户余额及项目筹集金额

\begin{center} \fbox{\parbox{0.8\textwidth}{TC-D012测试结果: 通过}} \end{center} \end{framed}

 \begin{framed} \textbf{测试用例编号:} TC-D013

\textbf{测试系统:} 公益课程系统

\textbf{测试类: } 授课人

\textbf{测试操作:} 创建课程

\textbf{重要级别:} 高

\textbf{测试场景:} 授课人已经登录到系统并进入创建课程页面

\textbf{测试内容:} 输入无效的课程信息,如空标题或空描述 输入合法的课程信息,但未上传课程封面 输入合法的课程信息,并上传了课程封面

\textbf{测试标准:} 系统应该提示错误信息并要求重新输入课程信息 系统应该提示错误信息并要求上传课程封面 系统应该成功创建课程,并在系统中显示课程信息及封面

\begin{center} \fbox{\parbox{0.8\textwidth}{TC-D013测试结果: 通过}} \end{center} \end{framed}

\begin{framed} \textbf{测试用例编号:} TC-D014

\textbf{测试系统:} 公益课程系统

\textbf{测试类: } 授课人

\textbf{测试操作:} 修改课程

\textbf{重要级别:} 中

\textbf{测试场景:} 授课人已经登录到系统并进入课程管理页面

\textbf{测试内容:}  选择不存在的课程进行修改 选择已结束的课程进行修改 选择未结束的课程进行修改
\textbf{测试标准:}系统应该提示错误信息并要求重新选择课程 系统应该提示错误信息并要求重新选择课程 系统应该成功修改课程信息,并在系统中更新课程信息及封面

\begin{center} \fbox{\parbox{0.8\textwidth}{TC-D014测试结果: 通过}} \end{center} \end{framed}

\begin{framed} \textbf{测试用例编号:} TC-D015

\textbf{测试系统:} 公益课程系统

\textbf{测试类: } 授课人

\textbf{测试操作:} 删除课程

\textbf{重要级别:} 高

\textbf{测试场景:} 授课人已经登录到系统并进入课程管理页面

\textbf{测试内容:} 选择不存在的课程进行删除,选择已结束的课程进行删除,选择未结束的课程进行删除

\textbf{测试标准:} 系统应该提示错误信息并要求重新选择课程,系统应该提示错误信息并要求重新选择课程,系统应该成功删除课程,并在系统中移除课程信息及封面 

\begin{center} \fbox{\parbox{0.8\textwidth}{TC-D015测试结果: 通过}} \end{center} \end{framed}

\paragraph{类行为测试}~{}
\begin{framed} \textbf{测试用例编号:} TC-E001

\textbf{测试系统:} 信息管理系统

\textbf{重要级别:} 高

\textbf{测试场景:} 用户管理

\textbf{测试内容:} 测试管理员添加用户功能是否正常,包括输入用户信息、设置用户权限等。

\textbf{覆盖状态迁移:} 用户添加成功或失败。

\textbf{测试标准:} 添加用户后,用户列表中应该出现新添加的用户信息。

\begin{center} \fbox{\parbox{0.8\textwidth}{TC-E001测试结果: 通过}} \end{center} \end{framed}

\begin{framed} \textbf{测试用例编号:} TC-E002

\textbf{测试系统:} 信息管理系统

\textbf{重要级别:} 中

\textbf{测试场景:} 活动管理

\textbf{测试内容:} 测试管理员发布活动功能是否正常,包括输入活动信息、设置活动时间、地点等。

\textbf{覆盖状态迁移:} 活动发布成功或失败。

\textbf{测试标准:} 发布活动后,活动列表中应该出现新发布的活动信息。

\begin{center} \fbox{\parbox{0.8\textwidth}{TC-E002测试结果: 通过}} \end{center} \end{framed}

\begin{framed} \textbf{测试用例编号:} TC-E003

\textbf{测试系统:} 信息管理系统

\textbf{重要级别:} 低

\textbf{测试场景:} 统计报表

\textbf{测试内容:} 测试管理员查看志愿服务统计报表功能是否正常,包括查看用户活动参与情况、活动完成情况等。

\textbf{覆盖状态迁移:} 查看报表成功或失败。

\textbf{测试标准:} 查看报表后,应该能够看到统计数据和图表。

\begin{center} \fbox{\parbox{0.8\textwidth}{TC-E003测试结果: 通过}} \end{center} \end{framed}

\begin{framed} \textbf{测试用例编号:} TC-E004

\textbf{测试系统:} 志愿服务系统

\textbf{重要级别:} 高

\textbf{测试场景:} 活动报名

\textbf{测试内容:} 测试用户活动报名功能是否正常,包括选择活动、填写报名信息等。

\textbf{覆盖状态迁移:} 报名成功或失败。

\textbf{测试标准:} 报名成功后,用户的报名信息应该出现在活动报名列表中。

\begin{center} \fbox{\parbox{0.8\textwidth}{TC-E004测试结果: 通过}} \end{center} \end{framed}

\begin{framed} \textbf{测试用例编号:} TC-E005

\textbf{测试系统:} 志愿服务系统

\textbf{重要级别:} 中

\textbf{测试场景:} 活动评价

\textbf{测试内容:} 测试用户对活动进行评价功能是否正常,包括选择活动、填写评价内容、评价星级等。

\textbf{覆盖状态迁移:} 评价成功或失败。

\textbf{测试标准:} 评价成功后,用户的评价信息应该出现在活动评价列表中。

\begin{center} \fbox{\parbox{0.8\textwidth}{TC-E005测试结果: 通过}} \end{center} \end{framed}

\begin{framed} \textbf{测试用例编号:} TC-E006

\textbf{测试系统:} 志愿服务系统

\textbf{重要级别:} 低

\textbf{测试场景:} 活动搜索

\textbf{测试内容:} 测试用户搜索活动功能是否正常,包括输入关键词、选择活动类型等。

\textbf{覆盖状态迁移:} 搜索成功或失败。

\textbf{测试标准:} 搜索成功后,应该能够看到与关键词相关的活动列表。

\begin{center} \fbox{\parbox{0.8\textwidth}{TC-E006测试结果: 通过}} \end{center} \end{framed}

\begin{framed} \textbf{测试用例编号:} TC-E007

\textbf{测试系统:} 爱心捐助系统

\textbf{重要级别:} 高

\textbf{测试场景:} 捐款

\textbf{测试内容:} 测试用户捐款功能是否正常,包括选择捐款项目、填写捐款金额等。

\textbf{覆盖状态迁移:} 捐款成功或失败。

\textbf{测试标准:} 捐款成功后,用户的捐款信息应该出现在捐款列表中。

\begin{center} \fbox{\parbox{0.8\textwidth}{TC-E007测试结果: 通过}} \end{center} \end{framed}

\begin{framed} \textbf{测试用例编号:} TC-E008

\textbf{测试系统:} 爱心捐助系统

\textbf{重要级别:} 中

\textbf{测试场景:} 捐款记录查询

\textbf{测试内容:} 测试用户查询捐款记录功能是否正常,包括输入查询条件、选择查询方式等。

\textbf{覆盖状态迁移:} 查询成功或失败。

\textbf{测试标准:} 查询成功后,应该能够看到与查询条件相关的捐款记录列表。

\begin{center} \fbox{\parbox{0.8\textwidth}{TC-E008测试结果: 通过}} \end{center} \end{framed}

\begin{framed} \textbf{测试用例编号:} TC-E009

\textbf{测试系统:} 爱心捐助系统

\textbf{重要级别:} 低

\textbf{测试场景:} 捐款统计

\textbf{测试内容:} 测试管理员查看捐款统计功能是否正常,包括查看捐款总额、捐款人数等。

\textbf{覆盖状态迁移:} 查看成功或失败。

\textbf{测试标准:} 查看成功后,应该能够看到捐款的相关统计数据和图表。

\begin{center} \fbox{\parbox{0.8\textwidth}{TC-E009测试结果: 通过}} \end{center} \end{framed}

\begin{framed} \textbf{测试用例编号:} TC-E010

\textbf{测试系统:} 公益课程系统

\textbf{重要级别:} 高

\textbf{测试场景:} 课程报名

\textbf{测试内容:} 测试用户课程报名功能是否正常,包括选择课程、填写报名信息等。

\textbf{覆盖状态迁移:} 报名成功或失败。

\textbf{测试标准:} 报名成功后,用户的报名信息应该出现在课程报名列表中。

\begin{center} \fbox{\parbox{0.8\textwidth}{TC-E010测试结果: 通过}} \end{center} \end{framed}

\begin{framed} \textbf{测试用例编号:} TC-E011

\textbf{测试系统:} 公益课程系统

\textbf{重要级别:} 中

\textbf{测试场景:} 课程评价

\textbf{测试内容:} 测试用户对课程进行评价功能是否正常,包括选择课程、填写评价内容、评价星级等。

\textbf{覆盖状态迁移:} 评价成功或失败。

\textbf{测试标准:} 评价成功后,用户的评价信息应该出现在课程评价列表中。

\begin{center} \fbox{\parbox{0.8\textwidth}{TC-E011测试结果: 通过}} \end{center} \end{framed}

\begin{framed} \textbf{测试用例编号:} TC-E012

\textbf{测试系统:} 公益课程系统

\textbf{重要级别:} 低

\textbf{测试场景:} 课程搜索

\textbf{测试内容:} 测试用户搜索课程功能是否正常,包括输入关键词、选择课程类型等。

\textbf{覆盖状态迁移:} 搜索成功或失败。

\textbf{测试标准:} 搜索成功后,应该能够看到与关键词相关的课程列表。

\begin{center} \fbox{\parbox{0.8\textwidth}{TC-E012测试结果: 通过}} \end{center} \end{framed}

\begin{framed} \textbf{测试用例编号:} TC-E013

\textbf{测试系统:} 交流论坛系统

\textbf{重要级别:} 高

\textbf{测试场景:} 发帖

\textbf{测试内容:} 测试用户发帖功能是否正常,包括输入帖子标题、内容等。

\textbf{覆盖状态迁移:} 发帖成功或失败。

\textbf{测试标准:} 发帖成功后,帖子应该出现在帖子列表中。

\begin{center} \fbox{\parbox{0.8\textwidth}{TC-E013测试结果: 通过}} \end{center} \end{framed}

\begin{framed} \textbf{测试用例编号:} TC-E014

\textbf{测试系统:} 交流论坛系统

\textbf{重要级别:} 中

\textbf{测试场景:} 回帖

\textbf{测试内容:} 测试用户回帖功能是否正常,包括选择帖子、输入回帖内容等。

\textbf{覆盖状态迁移:} 回帖成功或失败。

\textbf{测试标准:} 回帖成功后,回帖应该出现在帖子的回帖列表中。

\begin{center} \fbox{\parbox{0.8\textwidth}{TC-E014测试结果: 通过}} \end{center} \end{framed}

\begin{framed} \textbf{测试用例编号:} TC-E015

\textbf{测试系统:} 交流论坛系统

\textbf{重要级别:} 低

\textbf{测试场景:} 帖子搜索

\textbf{测试内容:} 测试用户搜索帖子功能是否正常,包括输入关键词、选择帖子类型等。

\textbf{覆盖状态迁移:} 搜索成功或失败。

\textbf{测试标准:} 搜索成功后,应该能够看到与关键词相关的帖子列表。

\begin{center} \fbox{\parbox{0.8\textwidth}{TC-E015测试结果: 通过}} \end{center} \end{framed}

\begin{framed} \textbf{测试用例编号:} TC-E016

\textbf{测试系统:} 志愿交友系统

\textbf{重要级别:} 高

\textbf{测试场景:} 用户登录

\textbf{测试内容:} 用户使用正确的用户名和密码登录志愿交友系统

\textbf{覆盖状态迁移:} 从未登录状态到已登录状态

\textbf{测试标准:} 用户成功登录系统,系统显示用户的个人信息和相关功能菜单。

\begin{center} \fbox{\parbox{0.8\textwidth}{TC-E016测试结果: 通过}} \end{center} \end{framed}

\begin{framed} \textbf{测试用例编号:} TC-E017

\textbf{测试系统:} 志愿交友系统

\textbf{重要级别:} 中

\textbf{测试场景:} 用户浏览志愿者列表

\textbf{测试内容:} 用户进入志愿者列表页面,查看志愿者信息

\textbf{覆盖状态迁移:} 从志愿者列表页面到志愿者详情页面

\textbf{测试标准:} 用户能够顺利进入志愿者列表页面,并能够查看志愿者的基本信息,如姓名、性别、年龄、所在地等,并能够顺利进入志愿者详情页面。

\begin{center} \fbox{\parbox{0.8\textwidth}{TC-E017测试结果: 通过}} \end{center} \end{framed}

\begin{framed} \textbf{测试用例编号:} TC-E018

\textbf{测试系统:} 志愿交友系统

\textbf{重要级别:} 高

\textbf{测试场景:} 用户发送私信

\textbf{测试内容:} 用户进入志愿者详情页面,点击“发送私信”按钮,填写私信内容并发送

\textbf{覆盖状态迁移:} 从志愿者详情页面到私信发送成功页面

\textbf{测试标准:} 用户能够顺利进入志愿者详情页面,并能够点击“发送私信”按钮,填写私信内容并成功发送。系统能够提示用户私信发送成功。

\begin{center} \fbox{\parbox{0.8\textwidth}{TC-E018测试结果: 通过}} \end{center} \end{framed}
\subsubsection{类间测试}
\begin{framed} \textbf{测试用例编号:} TC-F001

\textbf{测试系统:} 信息管理系统

\textbf{重要级别:} 高

\textbf{相关类:} 志愿者、系统管理员

\textbf{测试内容:} 测试当系统管理员添加志愿者时,志愿者类是否能够正确处理该消息并将志愿者添加到系统中。

\textbf{消息序列:} 系统管理员 -> 控制器 -> 志愿者

\textbf{测试标准:} 测试结果应该是志愿者类能够正确处理系统管理员添加志愿者的消息,并将志愿者添加到系统中。

\begin{center} \fbox{\parbox{0.8\textwidth}{TC-F001测试结果: 通过}} \end{center} \end{framed}

\begin{framed} \textbf{测试用例编号:} TC-F002

\textbf{测试系统:} 信息管理系统

\textbf{重要级别:} 高

\textbf{相关类:} 志愿者、系统管理员

\textbf{测试内容:} 测试当系统管理员删除志愿者时,志愿者类是否能够正确处理该消息并将志愿者从系统中移除。

\textbf{消息序列:} 系统管理员 -> 控制器 -> 志愿者

\textbf{测试标准:} 测试结果应该是志愿者类能够正确处理系统管理员删除志愿者的消息,并将志愿者从系统中移除。

\begin{center} \fbox{\parbox{0.8\textwidth}{TC-F002测试结果: 通过}} \end{center} \end{framed}

\begin{framed} \textbf{测试用例编号:} TC-F003

\textbf{测试系统:} 信息管理系统

\textbf{重要级别:} 高

\textbf{相关类:} 志愿者、系统管理员

\textbf{测试内容:} 测试当志愿者修改个人信息时,系统管理员是否能够正确处理该消息并更新志愿者的信息。

\textbf{消息序列:} 志愿者 -> 控制器 -> 系统管理员

\textbf{测试标准:} 测试结果应该是系统管理员能够正确处理志愿者修改个人信息的消息,并更新志愿者的信息。

\begin{center} \fbox{\parbox{0.8\textwidth}{TC-F003测试结果: 通过}} \end{center} \end{framed}

\begin{framed}
\textbf{测试用例编号:} TC-F004

\textbf{测试系统:} 志愿服务系统

\textbf{重要级别:} 高

\textbf{相关类:} 志愿者、志愿团队

\textbf{测试内容:} 测试当志愿者申请加入志愿团队时,志愿团队是否会收到加入请求

\textbf{消息序列:} 志愿者 -> 志愿团队

\textbf{测试标准:} 志愿团队收到志愿者的加入请求,并能够进行审核和处理

\begin{center}
\fbox{\parbox{0.8\textwidth}{TC-F004测试结果: 通过}}
\end{center}
\end{framed}

\begin{framed}
\textbf{测试用例编号:} TC-F005

\textbf{测试系统:} 志愿服务系统

\textbf{重要级别:} 高

\textbf{相关类:} 志愿者、志愿团队

\textbf{测试内容:} 测试志愿者退出志愿团队时,志愿团队是否能够正常处理

\textbf{消息序列:} 志愿者 -> 志愿团队

\textbf{测试标准:} 志愿团队能够从志愿团队成员名单中移除该志愿者,并将该志愿者的状态更新为“已退出”

\begin{center}
\fbox{\parbox{0.8\textwidth}{TC-F005测试结果: 通过}}
\end{center}
\end{framed}

\begin{framed} \textbf{测试用例编号:} TC-F006

\textbf{测试系统:} 爱心捐助系统

\textbf{重要级别:} 高

\textbf{相关类:} 捐款者、志愿团队

\textbf{测试内容:} 测试当捐款者捐款时,志愿团队是否能够收到捐款通知

\textbf{消息序列:} 捐款者 -> 爱心捐助系统 -> 爱心捐助系统 -> 志愿团队

\textbf{测试标准:} 志愿团队能够在收到捐款通知后及时处理并向捐款者发送感谢信

\begin{center} \fbox{\parbox{0.8\textwidth}{TC-F006测试结果: 通过}} \end{center} \end{framed}

\begin{framed} \textbf{测试用例编号:} TC-F007

\textbf{测试系统:} 爱心捐助系统

\textbf{重要级别:} 高

\textbf{相关类:} 捐款者、志愿团队

\textbf{测试内容:} 测试当捐款者取消捐款时,志愿团队是否能够正常处理

\textbf{消息序列:} 捐款者 -> 爱心捐助系统 -> 爱心捐助系统 -> 志愿团队

\textbf{测试标准:} 志愿团队能够从捐款者名单中移除该捐款者,并将该捐款者的状态更新为“已取消”

\begin{center} \fbox{\parbox{0.8\textwidth}{TC-F007测试结果: 通过}} \end{center} \end{framed}

\begin{framed} \textbf{测试用例编号:} TC-F008

\textbf{测试系统:} 公益课程系统

\textbf{重要级别:} 高

\textbf{相关类:} 志愿者、授课人

\textbf{测试内容:} 测试当志愿者申请成为授课人时,授课人类是否能够正确处理该请求。

\textbf{消息序列:} 志愿者 -> 控制器 -> 授课人

\textbf{测试标准:} 测试结果应该是授课人类能够正确处理志愿者申请成为授课人的请求,并将志愿者添加到授课人列表中。

\begin{center} \fbox{\parbox{0.8\textwidth}{TC-F008测试结果: 通过}} \end{center} \end{framed}

\begin{framed} \textbf{测试用例编号:} TC-F009

\textbf{测试系统:} 公益课程系统

\textbf{重要级别:} 高

\textbf{相关类:} 志愿者、授课人

\textbf{测试内容:} 测试当授课人发布课程时,志愿者类是否能够正确处理该消息并接收课程信息。

\textbf{消息序列:} 授课人 -> 控制器 -> 志愿者

\textbf{测试标准:} 测试结果应该是志愿者类能够正确处理授课人发布课程的消息,并将课程信息添加到志愿者的课程列表中。

\begin{center} \fbox{\parbox{0.8\textwidth}{TC-F009测试结果: 通过}} \end{center} \end{framed}

\begin{framed} \textbf{测试用例编号:} TC-F010

\textbf{测试系统:} 交流论坛系统

\textbf{重要级别:} 高

\textbf{相关类:} 志愿者、系统管理员

\textbf{测试内容:} 测试当志愿者发表帖子时,系统管理员是否能够正确处理该消息并审核帖子。

\textbf{消息序列:} 志愿者 -> 控制器 -> 系统管理员

\textbf{测试标准:} 测试结果应该是系统管理员能够正确处理志愿者发表帖子的消息,并审核帖子内容是否符合规范。审核通过后,帖子应该被发布到论坛上。

\begin{center} \fbox{\parbox{0.8\textwidth}{TC-F010测试结果: 通过}} \end{center} \end{framed}

\begin{framed} \textbf{测试用例编号:} TC-F011

\textbf{测试系统:} 交流论坛系统

\textbf{重要级别:} 高

\textbf{相关类:} 志愿者、系统管理员

\textbf{测试内容:} 测试当系统管理员删除帖子时,志愿者是否能够正确处理该消息并删除相应的帖子。

\textbf{消息序列:} 系统管理员 -> 控制器 -> 志愿者

\textbf{测试标准:} 测试结果应该是志愿者能够正确处理系统管理员删除帖子的消息,并删除相应的帖子。帖子应该从论坛中移除,不再显示给其他用户。

\begin{center} \fbox{\parbox{0.8\textwidth}{TC-F011测试结果: 通过}} \end{center} \end{framed}
\subsubsection{基于场景的测试}
\begin{framed} \textbf{测试用例编号:} TC-G001

\textbf{测试系统:} 信息管理系统

\textbf{重要级别:} 高

\textbf{测试场景:} 管理员添加新的志愿者信息

\textbf{测试标准:} 管理员能够成功添加新的志愿者信息,并且信息能够被正确保存在系统中。

\begin{center} \fbox{\parbox{0.8\textwidth}{TC-G001测试结果: 通过}} \end{center} \end{framed}

\begin{framed} \textbf{测试用例编号:} TC-G002

\textbf{测试系统:} 信息管理系统

\textbf{重要级别:} 中

\textbf{测试场景:} 志愿者修改个人信息

\textbf{测试标准:} 志愿者能够成功修改个人信息,并且信息能够被正确保存在系统中。

\begin{center} \fbox{\parbox{0.8\textwidth}{TC-G002测试结果: 通过}} \end{center} \end{framed}

\begin{framed} \textbf{测试用例编号:} TC-G003

\textbf{测试系统:} 志愿服务系统

\textbf{重要级别:} 高

\textbf{测试场景:} 志愿者申请参加某项志愿服务活动

\textbf{测试标准:} 志愿者能够成功申请参加某项志愿服务活动,并且申请信息能够被正确保存在系统中。

\begin{center} \fbox{\parbox{0.8\textwidth}{TC-G003测试结果: 通过}} \end{center} \end{framed}

\begin{framed} \textbf{测试用例编号:} TC-G004

\textbf{测试系统:} 志愿服务系统

\textbf{重要级别:} 中

\textbf{测试场景:} 志愿者查看自己已申请的志愿服务活动

\textbf{测试标准:} 志愿者能够成功查看自己已申请的志愿服务活动,并且信息能够被正确展示在系统中。

\begin{center} \fbox{\parbox{0.8\textwidth}{TC-G004测试结果: 通过}} \end{center} \end{framed}

\begin{framed} \textbf{测试用例编号:} TC-G005

\textbf{测试系统:} 爱心捐助系统

\textbf{重要级别:} 高

\textbf{测试场景:} 用户捐款

\textbf{测试标准:} 用户能够成功进行捐款,并且捐款信息能够被正确保存在系统中。

\begin{center} \fbox{\parbox{0.8\textwidth}{TC-G005测试结果: 通过}} \end{center} \end{framed}

\begin{framed} \textbf{测试用例编号:} TC-G006

\textbf{测试系统:} 爱心捐助系统

\textbf{重要级别:} 中

\textbf{测试场景:} 管理员查看捐款记录

\textbf{测试标准:} 管理员能够成功查看捐款记录,并且记录能够被正确展示在系统中。

\begin{center} \fbox{\parbox{0.8\textwidth}{TC-G006测试结果: 通过}} \end{center} \end{framed}

\begin{framed} \textbf{测试用例编号:} TC-G007

\textbf{测试系统:} 公益课程系统

\textbf{重要级别:} 高

\textbf{测试场景:} 用户报名参加公益课程

\textbf{测试标准:} 用户能够成功报名参加公益课程,并且报名信息能够被正确保存在系统中。

\begin{center} \fbox{\parbox{0.8\textwidth}{TC-G007测试结果: 通过}} \end{center} \end{framed}

\begin{framed} \textbf{测试用例编号:} TC-G008

\textbf{测试系统:} 公益课程系统

\textbf{重要级别:} 中

\textbf{测试场景:} 管理员添加新的公益课程

\textbf{测试标准:} 管理员能够成功添加新的公益课程,并且信息能能够被正确保存在系统中。

\begin{center} \fbox{\parbox{0.8\textwidth}{TC-G008测试结果: 通过}} \end{center} \end{framed}

\begin{framed} \textbf{测试用例编号:} TC-G009

\textbf{测试系统:} 交流论坛系统

\textbf{重要级别:} 高

\textbf{测试场景:} 用户发表帖子

\textbf{测试标准:} 用户能够成功发表帖子,并且帖子能够被正确保存在系统中。

\begin{center} \fbox{\parbox{0.8\textwidth}{TC-G009测试结果: 通过}} \end{center} \end{framed}

\begin{framed} \textbf{测试用例编号:} TC-G010

\textbf{测试系统:} 交流论坛系统

\textbf{重要级别:} 中

\textbf{测试场景:} 用户回复帖子

\textbf{测试标准:} 用户能够成功回复帖子,并且回复信息能够被正确保存在系统中。

\begin{center} \fbox{\parbox{0.8\textwidth}{TC-G010测试结果: 通过}} \end{center} \end{framed}

\begin{framed} \textbf{测试用例编号:} TC-G011

\textbf{测试系统:} 志愿交友系统

\textbf{重要级别:} 高

\textbf{测试场景:} 用户创建个人资料

\textbf{测试标准:} 用户能够成功创建个人资料,并且资料能够被正确保存在系统中。

\begin{center} \fbox{\parbox{0.8\textwidth}{TC-G011测试结果: 通过}} \end{center} \end{framed}

\begin{framed} \textbf{测试用例编号:} TC-G012

\textbf{测试系统:} 志愿交友系统

\textbf{重要级别:} 中

\textbf{测试场景:} 用户查看其他用户的个人资料

\textbf{测试标准:} 用户能够成功查看其他用户的个人资料,并且资料能够被正确展示在系统中。

\begin{center} \fbox{\parbox{0.8\textwidth}{TC-G012测试结果: 通过}} \end{center} \end{framed}

\begin{framed} \textbf{测试用例编号:} TC-G013

\textbf{测试系统:} 志愿交友系统

\textbf{重要级别:} 高

\textbf{测试场景:} 用户搜索其他用户

\textbf{测试标准:} 用户能够成功搜索到其他用户,并且搜索结果能够被正确展示在系统中。

\begin{center} \fbox{\parbox{0.8\textwidth}{TC-G013测试结果: 通过}} \end{center} \end{framed}

\begin{framed} \textbf{测试用例编号:} TC-G014

\textbf{测试系统:} 志愿交友系统

\textbf{重要级别:} 中

\textbf{测试场景:} 用户发送私信

\textbf{测试标准:} 用户能够成功发送私信,并且私信能够被正确保存在系统中。

\begin{center} \fbox{\parbox{0.8\textwidth}{TC-G014测试结果: 通过}} \end{center} \end{framed}

\begin{framed} \textbf{测试用例编号:} TC-G015

\textbf{测试系统:} 志愿交友系统

\textbf{重要级别:} 高

\textbf{测试场景:} 用户删除私信

\textbf{测试标准:} 用户能够成功删除私信,并且私信能够被正确从系统中移除。

\begin{center} \fbox{\parbox{0.8\textwidth}{TC-G015测试结果: 通过}} \end{center} \end{framed}
\newpage