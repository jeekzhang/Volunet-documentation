\begin{table}[H]  
\caption{“登录权限查询”加工词条描述}  
\begin{center}  
    \begin{tabular}{l p{11cm}} 
        \hline
        \quad 名称: & 登录权限查询 \\
        \hline
        \quad 编号: & 4.1.1 \\
        \hline
        \quad 简述: & 查询用户登录权限的功能 \\
        \hline
        \quad 输入: & 授课申请书、已登录用户权限、用户ID \\
        \hline
        \quad 输出: & 登录状态、权限不足提示、用户名 \\
        \hline
        \quad 逻辑: & 查询对应用户ID的权限信息。 \\
        \hline
    \end{tabular}
    \label{tab1}
\end{center}
\end{table}


\begin{algorithm}[H] 
    \renewcommand{\thealgorithm}{}
    \caption{“登录权限查询”加工小说明} 
    \label{alg3} 
    \begin{algorithmic}[1]
        \STATE Get 系统时间 As 查询时间
        \STATE Write 账号 + 密码 + 登录状态  To 登录权限查询信息 
    \end{algorithmic} 
\end{algorithm}

\begin{table}[H]  
\caption{“授课申请处理”加工词条描述}  
\begin{center}  
    \begin{tabular}{l p{11cm}} 
        \hline
        \quad 名称: & 授课申请处理 \\
        \hline
        \quad 编号: & 4.1.2 \\
        \hline
        \quad 简述: & 处理授课申请的功能 \\
        \hline
        \quad 输入: & 授课申请书、用户ID \\
        \hline
        \quad 输出: & 申请通过提示、申请处理时间、课程名、课程信息 \\
        \hline
        \quad 逻辑: & 通过对授课申请书处理申请、建立课程信息 \\
        \hline
    \end{tabular}
    \label{tab1}
\end{center}
\end{table}


\begin{algorithm}[H]
    \renewcommand{\thealgorithm}{}
    \caption{“授课申请处理”加工小说明} 
    \label{alg3} 
    \begin{algorithmic}[1]
        \IF{授课申请处理通过} 
        \STATE Write 用户ID to 授课人信息 
        \STATE Write {理由} + 反应时间  To 申请通过提示
        \ELSE
        \STATE Write {理由} + 反应时间  To 授课申请处理反馈
        \ENDIF 
    \end{algorithmic} 
\end{algorithm}


\begin{table}[H]  
\caption{“课程内容设置”加工词条描述}  
\begin{center}  
    \begin{tabular}{l p{11cm}} 
        \hline
        \quad 名称: & 课程内容设置 \\
        \hline
        \quad 编号: & 4.1.3 \\
        \hline
        \quad 简述: & 设置课程内容的功能 \\
        \hline
        \quad 输入: & 课程信息、用户ID、用户权限 \\
        \hline 
        \quad 输出: & 课程信息、课程资料 \\
        \hline
        \quad 逻辑: & 根据学科、用户需求等因素,确定课程的内容和教学目标,以及相应的评价方式。 \\
        \hline
    \end{tabular}
    \label{tab1}
\end{center}
\end{table}


\begin{algorithm}[H]
    \renewcommand{\thealgorithm}{}
    \caption{“课程内容设置”加工小说明} 
    \label{alg3} 
    \begin{algorithmic}[1]
        \STATE Get 课程名 + 课程类别 From 可选课程信息
        \STATE Insert 课程测试 + 课程章节 Into 课程资料
    \end{algorithmic} 
\end{algorithm}

\begin{table}[H]  
\caption{“课程内容修改”加工词条描述}  
\begin{center}  
    \begin{tabular}{l p{11cm}} 
        \hline
        \quad 名称: & 课程内容修改 \\
        \hline
        \quad 编号: & 4.1.4 \\
        \hline
        \quad 简述: & 授课人修改课程内容的功能 \\
        \hline
        \quad 输入: & 课程信息修改单+用户ID+用户权限 \\
        \hline
        \quad 输出: & 修改后的课程信息 \\
        \hline
        \quad 逻辑: & 授课人通过上传修改单现有课程内容进行更新、改进或调整。 \\
        \hline
    \end{tabular}
    \label{tab1}
\end{center}
\end{table}


\begin{algorithm}[H] 
    \renewcommand{\thealgorithm}{}
    \caption{“课程内容修改”加工小说明} 
    \label{alg3} 
    \begin{algorithmic}[1]
        \STATE Get 课程名 From 课程信息修改单
        \STATE Update Item In 课程内容 Match 课程名 With 课程内容修改信息
    \end{algorithmic} 
\end{algorithm}

\begin{table}[H]  
\caption{“授课证书颁发”加工词条描述}  
\begin{center}  
    \begin{tabular}{l p{11cm}} 
        \hline
        \quad 名称: & 授课证书颁发 \\
        \hline
        \quad 编号: & 4.1.5 \\
        \hline
        \quad 简述: & 颁发授课人授课证书的功能 \\
        \hline
        \quad 输入: & 课程信息、用户名 \\
        \hline
        \quad 输出: & 授课证书 \\
        \hline
        \quad 逻辑: & 由相关机构颁发给授课人的证明授课能力和教学水平的证书。 \\
        \hline
    \end{tabular}
    \label{tab1}
\end{center}
\end{table}


\begin{algorithm}[H]
    \renewcommand{\thealgorithm}{}
    \caption{“授课证书颁发”加工小说明} 
    \label{alg3} 
    \begin{algorithmic}[1]
        \STATE Analyze 授课资质
        \STATE Generate 授课人要求条件 Based On 授课人信息
        \STATE Write 授课准许 To 授课证书
    \end{algorithmic} 
\end{algorithm}

\begin{table}[H]  
\caption{“学生上课情况分析”加工词条描述}  
\begin{center}  
    \begin{tabular}{l p{11cm}} 
        \hline
        \quad 名称: & 学生上课情况分析 \\
        \hline
        \quad 编号: & 4.1.6 \\
        \hline
        \quad 简述: & 输出分析的学生上课情况的功能 \\
        \hline
        \quad 输入: & 学生修读情况、授课人名 \\
        \hline
        \quad 输出: & 修读情况分析、授课人名 \\
        \hline
        \quad 逻辑: & 对学生在上课过程中的表现和情况进行分析和评估,指导教学和促进学生学习。 \\
        \hline
    \end{tabular}
    \label{tab1}
\end{center}
\end{table}



\begin{algorithm}[H]
    \renewcommand{\thealgorithm}{}
    \caption{“学生上课情况分析”加工小说明} 
    \label{alg3} 
    \begin{algorithmic}[1]
        \STATE Analyze 学生修读情况
        \FOR{ 用户ID + 课程进度 in 学生上课情况}
        \STATE Select 对应课程内容 From 课程信息 Match 用户ID
        \STATE Add 对应测试成绩+用户ID To 修读情况分析
        \ENDFOR
    \end{algorithmic} 
\end{algorithm}